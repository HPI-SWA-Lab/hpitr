% \iffalse meta-comment
% ============================================================================
% hpitr.dtx
% Copyright (c) Daniel Richter, Uwe Hentschel 2013-2014
% Copyright (c) Tobias Pape, 2014
%
% This work may be distributed and/or modified under the conditions of
% the LaTeX Project Public License, version 1.3c of the license.
% The latest version of this license is in
%   http://www.latex-project.org/lppl.txt
% and version 1.3c or later is part of all distributions of LaTeX 
% version 2005/12/01 or later.
%
% This work has the LPPL maintenance status "maintained".
%
% The Current Maintainer and author of this work is Tobias Pape.
%
% This work consists of the file `hpitr.dtx'.
%
%^^A ==========================================================================
%^^A ==========================================================================
%^^A ==========================================================================
% \fi^^A meta-comment
% \CharacterTable
%  {Upper-case    \A\B\C\D\E\F\G\H\I\J\K\L\M\N\O\P\Q\R\S\T\U\V\W\X\Y\Z
%   Lower-case    \a\b\c\d\e\f\g\h\i\j\k\l\m\n\o\p\q\r\s\t\u\v\w\x\y\z
%   Digits        \0\1\2\3\4\5\6\7\8\9
%   Exclamation   \!     Double quote  \"     Hash (number) \#
%   Dollar        \$     Percent       \%     Ampersand     \&
%   Acute accent  \'     Left paren    \(     Right paren   \)
%   Asterisk      \*     Plus          \+     Comma         \,
%   Minus         \-     Point         \.     Solidus       \/
%   Colon         \:     Semicolon     \;     Less than     \<
%   Equals        \=     Greater than  \>     Question mark \?
%   Commercial at \@     Left bracket  \[     Backslash     \\
%   Right bracket \]     Circumflex    \^     Underscore    \_
%   Grave accent  \`     Left brace    \{     Vertical bar  \|
%   Right brace   \}     Tilde         \~}
% \CheckSum{0}
%
% \iffalse meta-comment
%<*dtx|ins>
\expandafter\ifx\csname ProvidesFile\endcsname\relax
  \def\ProvidesFile#1[#2]{\message{#1: #2}}%
\fi
%</dtx|ins>
%<class>\NeedsTeXFormat{LaTeX2e}[1995/12/01]
%<class>\ProvidesClass{hpitr}%
%<*dtx|ins|doc|manual>
\ProvidesFile{%
  hpitr%
%<manual>  -manual%
%<*dtx>
  .dtx%
%</dtx>
%<ins>  .ins%
%<doc>  .ltx%
}%
%</dtx|ins|doc|manual>
%<*dtx|ins|doc|class|README>
%<README>hpitr
%<README>  Copyright (c) Tobias Pape <tobias.pape at hpi.de>
%<*!README>
[%
%</!README>
  2014/09/04 v0.7
%<!README>^^J
  A unified class for Technical Reports at the HPI
%<*!README>
]
%</!README>
%</dtx|ins|doc|class|README>
% \fi^^A meta-comment
% ^^A -------------------------------------------------------------------------
%
%
%
%
% \changes{v0.1}{2014/06/01}{Inital class derived from polze.cls}
% \changes{v0.2}{2014/07/10}{Minmial changes to play nice with babel and
% subcaption}
% \changes{v0.5}{2014/08/15}{Small fixes to abstract and title, loading of float}
% \changes{v0.6}{2014/08/27}{Provide \cs{keywords} command.}
% \changes{v0.7}{2014/09/04}{Discriminate between proceedings and collections}
%
% \GetFileInfo{hpitr.dtx}
%^^A ==========================================================================
%^^A ==========================================================================
%^^A ==========================================================================
%
%\makeatletter
% ^^A Heiko Oberdiek's fix from latexbugs (latex/3540):
%\begingroup
%  \def\x\begingroup#1\@nil{\endgroup
%    \def\DoNotIndex{\begingroup
%      \@tfor\@tempa:=\#\$\&\^\_\|\~\ \<\do{\expandafter\@makeother\@tempa}#1}}
%\expandafter\x\DoNotIndex\@nil
%\def\PercentIndex{}\def\LeftBraceIndex{}\def\RightBraceIndex{}
%
% ^^A Non-Index list derived from microtype.dtx and titlepage.dtx
%
%  \DoNotIndex{\!,\",\',\(,\),\*,\+,\,,\-,\.,\/,\:,\;,\<,\=,\>,\?,\[,\\,\],\`,
%  \#,\$,\&,\^,\_,\|,\~,\ ,\@Alph,\@M,
%  \advance,\afterassignment,\aftergroup,\begingroup,\bgroup,\catcode,\char,
%  \chardef,\csname,\def,\divide,\edef,\egroup,\else,\endcsname,\endgroup,
%  \endinput,\escapechar,\everypar,\expandafter,\fi,\futurelet,\gdef,\global,
%  \hbadness,\hbox,\hsize,\hskip,\if,\ifcase,\ifcat,\ifdim,\iffalse,\ifhbox,
%  \ifhmode,\ifmmode,\ifnum,\iftrue,\ifx,\immediate,\input,\inputlineno,
%  \jobname,\kern,\lastkern,\lastskip,\let,\lowercase,\maxdimen,\meaning,
%  \multiply,\newlinechar,\noexpand,\number,\or,\parfillskip,\pretolerance,
%  \relax,\setbox,\showboxdepth,\string,\the,\tolerance,\unkern,\unskip,
%  \uppercase,\vbox,\wd,\write,\xdef,\font,\fontdimen,\nullfont,\sfcode,
%  \spacefactor,\spaceskip,\xspaceskip}
%  \DoNotIndex{\detokenize,\dimexpr,\eTeXversion,\ifcsname,\ifdefined,\numexpr}
%  \DoNotIndex{\@backslashchar,\@cclv,\@cclvi,\@classoptionslist,\@currext,
%  \@currname,\@defaultunits,\@empty,\@expandtwoargs,\@firstofone,\@firstoftwo,
%  \@gobble,\@gobbletwo,\@ifl@aded,\@ifpackagelater,\@ifpackageloaded,\@ifstar,
%  \@ifundefined,\@let@token,\@m,\@makeother,\@minus,\@nameuse,\@ne,\@nil,\@nnil,
%  \@onelevel@sanitize,\@onlypreamble,\@percentchar,\@pkgextension,\@plus,
%  \@ptionlist,\@removeelement,\@secondoftwo,\@spaces,\@sptoken,\@tempa,\@tempb,
%  \@tempc,\@tempcnta,\@tempcntb,\@tempdima,\@typeset@protect,\@undefined,
%  \@unprocessedoptions,\@unusedoptionlist,\@xobeysp,\check@icr,\color@begingroup,
%  \color@endgroup,\g@addto@macro,\hmode@bgroup,\m@ne,\maybe@ic,\maybe@ic@,
%  \nfss@text,\not@math@alphabet,\on@line,\p@,\set@display@protect,\strip@prefix,
%  \strip@pt,\tw@,\z@,\z@skip,\zap@space,\active,\documentclass,\leavevmode,
%  \makeatletter,\mbox,\newcommand,\newcount,\newdimen,\newif,\newskip,
%  \newtoks,\nobreak,\nonfrenchspacing,\normalsize,\renewcommand,\space,
%  \AtBeginDocument,\AtEndOfPackage,\CheckCommand,\CurrentOption,
%  \DeclareRobustCommand,\IfFileExists,\InputIfFileExists,\MessageBreak,
%  \PackageError,\PackageInfo,\PackageWarning,\RequirePackage,
%  \@@enc@update,\cf@encoding,\f@encoding}
%  \DoNotIndex{\foreign@language,\languagename,\select@language,\shorthandoff}
%  \DoNotIndex{\pdfstringdefDisableCommands,\pdfstringdefWarn} ^^A hyperref
%  \DoNotIndex{\ifpdf} ^^A ifpdf
%
%  \DoNotIndex{\wd,\xdef,\year,\z@}
%  \DoNotIndex{\@abstrtfalse,\@abstrttrue,\@addtoreset,\@afterheading}
%  \DoNotIndex{\@afterindentfalse,\@alph,\@arabic}
%  \DoNotIndex{\@beginparpenalty}
%  \DoNotIndex{\@car,\@cdr,\@centercr}
%  \DoNotIndex{\@dblfloat,\@dotsep}
%  \DoNotIndex{\@dottedtocline,\@empty,\@endparpenalty}
%  \DoNotIndex{\@float,\@fontswitch}
%  \DoNotIndex{\@gobbletwo}
%  \DoNotIndex{\@hangfrom,\@highpenalty}
%  \DoNotIndex{\@ifnextchar,\@ifundefined,\@itempenalty}
%  \DoNotIndex{\@latex@warning}
%  \DoNotIndex{\@m,\@medpenalty,\@minus,\@mkboth,\@mparswitchfalse}
%  \DoNotIndex{\@mparswitchtrue}
%  \DoNotIndex{\@ne,\@nil,\@nobreakfalse,\@nobreaktrue,\@nomath}
%  \DoNotIndex{\@plus}
%  \DoNotIndex{\@tempa,\@tempcnta,\@tempdima,\@tempskipka}
%  \DoNotIndex{\@tempswafalse,\@tempswatrue}
%  \DoNotIndex{\@tempb,\@tempcntb,\@tempdimb,\@tempskipkb}
%  \DoNotIndex{\@tempswbfalse,\@tempswbtrue}
%  \DoNotIndex{\@tempc,\@tempcntc,\@tempdimc,\@tempskipkc}
%  \DoNotIndex{\@tocrmarg,\@topnewpage,\@topnum,\@twocolumnfalse}
%  \DoNotIndex{\@twocolumntrue,\@twosidefalse,\@twosidetrue}
%  \DoNotIndex{\@whiledim,\@whilenum}
%  \DoNotIndex{\addcontentsline,\addpenalty,\addtocontents,\addtolength}
%  \DoNotIndex{\addvspace,\advance}
%  \DoNotIndex{\begin,\begingroup,\bfseries,\box,\bullet}
%    \DoNotIndex{\c@figure,\c@page,\c@secnumdepth,\c@table,\c@tocdepth}
%    \DoNotIndex{\cdot,\centering,\changes,\cleardoublepage,\clearpage}
%    \DoNotIndex{\cmd,\col@number,\CurrentOption,\CodelineIndex,\cs}
%    \DoNotIndex{\day,\dblfloatpagefraction,\dbltopfraction}
%    \DoNotIndex{\DeclareOldFontCommand,\DeclareOption,\def,\DisableCrossrefs}
%    \DoNotIndex{\divide,\DoNotIndex}
%    \DoNotIndex{\ifdim,\else,\fi,\empty,\em,\EnableCrossrefs,\end}
%  \DoNotIndex{\end@dblfloat}
% \DoNotIndex{\end@float,\endgroup,\endlist,\endquotation,\endtitlepage}
% \DoNotIndex{\everypar,\ExecuteOptions,\expandafter}
% \DoNotIndex{\fboxrule,\fboxsep}
%  \DoNotIndex{\gdef,\global}
% \DoNotIndex{\hangindent,\hbox,\hfil,\hrule,\hsize,\hskip,\hspace,\hss}
% \DoNotIndex{\if@tempswa,\ifcase,\or,\fi,\fi}
% \DoNotIndex{\ifnum,\ifodd,\ifx,\fi,\fi,\fi}
% \DoNotIndex{\include,\input,\InputIfFileExists,\item,\itshape}
% \DoNotIndex{\kern,\leavevmode,\leftmark,\leftskip,\let,\lineskip}
% \DoNotIndex{\list,\long}
% \DoNotIndex{\m@ne,\m@th,\marginpar,\markboth,\markright,\mathbf,\mathcal}
% \DoNotIndex{\mathit,\mathnormal,\mathrm,\mathsf,\mathtt,\MessageBreak}
% \DoNotIndex{\month}
% \DoNotIndex{\newblock,\newcommand,\newcount,\newcounter,\newdimen}
% \DoNotIndex{\newenvironment,\newlength,\newpage,\nobreak,\noindent}
% \DoNotIndex{\normalfont,\normallineskip,\normalsize,\null,\number}
% \DoNotIndex{\numberline,\normalcolor}
% \DoNotIndex{\OldMakeindex,\OnlyDescription,\overfullrule}
% \DoNotIndex{\p@,\PackageError,\PackageInfo,\PackageWarningNoLine}
% \DoNotIndex{\pagenumbering,\pagestyle,\par,\paragraph,\parbox}
% \DoNotIndex{\PassOptionsToPackage,\pcal,\penalty,\pmit,\PrintChanges}
% \DoNotIndex{\PrintIndex,\ProcessOptions,\protect,\providecommand}
% \DoNotIndex{\ProvidesClass}
% \DoNotIndex{\raggedbottom,\raggedleft,\raggedright,\refstepcounter,\relax}
% \DoNotIndex{\renewcommand,\RequirePackage,\reset@font}
% \DoNotIndex{\rightmargin,\rightmark,\rightskip,\rmfamily,\@Roman,\@roman}
% \DoNotIndex{\scshape,\secdef,\setbox,\setcounter,\setlength}
% \DoNotIndex{\settowidth,\sfcode,\sffamily,\skip,\sloppy,\slshape,\space}
% \DoNotIndex{\string}
% \DoNotIndex{\the,\thispagestyle,\triangleright,\ttfamily}
% \DoNotIndex{\twocolumn,\typeout}
%  \DoNotIndex{\undefined,\usecounter}
% \DoNotIndex{\vfil,\vfill,\vspace}
%  \DoNotIndex{\ifthenelse,\OR,\AND,\boolean}
% \makeatother
%
%^^A ==========================================================================
%^^A ==========================================================================
%^^A ==========================================================================
% \StopEventually{
%  \newpage
%  \appendix
%  \PrintChanges
%  \PrintIndex
% }
%
% \section{Implementation}
% \label{sec:Implementation}
%
% \subsection{The Installation Driver `\protect\File{hpitr.ins}'}
% \label{sec:hpitr-ins}
% \iffalse meta-comment
%<*dtx|ins>
% \fi ^^A meta-comment
% First of all we produce \File{hpitr.ins}, the installation driver. It
% starts very common with loading \File{docstrip}, preamble declaration and
% start of generation.
%    \begin{macrocode}
\def\batchfile{hpitr.dtx}
\input docstrip.tex
\ifToplevel{%
  \Msg{********************************************************************}
  \Msg{*}
  \Msg{* Steps of hpitr generation:}
  \Msg{* ==========================}
  \Msg{*}
  \Msg{* - Generation of all needed files:}
  \keepsilent
  \askforoverwritefalse
}

\preamble
Copyright (c) Daniel Richter, Uwe Hentschel 2013-2014
Copyright (c) Tobias Pape, 2014

This file was generated from file(s) of hpitr distribution.
----------------------------------------------------------------------

This work may be distributed and/or modified under the conditions of
the LaTeX Project Public License, version 1.3c of the license.
The latest version of this license is in
  http://www.latex-project.org/lppl.txt
and version 1.3c or later is part of all distributions of LaTeX 
version 2005/12/01 or later.

This work has the LPPL maintenance status "maintained".

The Current Maintainer and author of this work is Tobias Pape.

This file may only be distributed together with the file
`hpitr.dtx'. You may however distribute the file `hpitr.dtx' 
without this file.
\endpreamble

\generate{%
%    \end{macrocode}
% But while the dtx file is the ins file itself, we will never generate this
% file.
% \begin{verbatim}
%  \file{hpitr.ins}{\from{hpitr.dtx}{ins}}% not needed
% \end{verbatim}\vskip-\baselineskip
%    \begin{macrocode}
  \file{hpitr.cls}{\from{hpitr.dtx}{class}}%
  \file{hpitr-manual.ltx}{\from{hpitr.dtx}{doc,manual}}%
  \file{hpitr.ltx}{\from{hpitr.dtx}{doc}}%
  \nopreamble\nopostamble
  \file{example-single.tex}{\from{hpitr.dtx}{example,single}}%
  \file{example-singlea.tex}{\from{hpitr.dtx}{example,single,article}}%
  \file{example-proceedings.tex}{\from{hpitr.dtx}{example,proceedings}}%
  \file{README.txt}{\from{hpitr.dtx}{README}}%
}%
%    \end{macrocode}
% After file generation, wie use shell escapes (aka \verb|\write18| feature) to
% create \File{hpitr-manual.pdf}, the manual, and \File{hpitr.pdf}, the full
% documentation.
%    \begin{macrocode}
\ifToplevel{%
  \Msg{* \space\space done.}
  \Msg{* - Generation of manual and examples:}
  \expandafter\ifx\csname pdfshellescape\endcsname\relax
    \Msg{* WARNING: \string\pdfshellescape\space not available!}
    \Msg{* WARNING: You should run this file with shell-escapes enabled.}
    \Msg{* WARNING: Otherwise you have to call}
    \Msg{* WARNING: \space\space lualatex hpitr.ltx}
    \Msg{* WARNING: to generate the manual and examples on your own!}
    \batchmode \errmessage{}\csname @@end\endcsname\csname end\endcsname
  \else\ifnum \pdfshellescape=1 \else
    \Msg{* WARNING: shell-escapes not activated!}
    \Msg{* WARNING: You should run this file with shell-escapes enabled.}
    \Msg{* WARNING: Otherwise you have to call}
    \Msg{* WARNING: \space\space lualatex hpitr.ltx}
    \Msg{* WARNING: to generate the manual and examples on your own!}
    \batchmode \errmessage{}\csname @@end\endcsname\csname end\endcsname
  \fi\fi
}

\immediate\write18{cp README.txt README || copy README.txt README}
\immediate\write18{lualatex -interaction=nonstopmode hpitr.ltx}
\immediate\write18{mkindex hpitr}
\immediate\write18{lualatex -interaction=nonstopmode hpitr.ltx}
\immediate\write18{mkindex hpitr}
\immediate\write18{lualatex -interaction=nonstopmode hpitr.ltx}

\immediate\write18{lualatex -interaction=nonstopmode hpitr-manual.ltx}
\immediate\write18{mkindex hpitr-manual}
\immediate\write18{lualatex -interaction=nonstopmode hpitr-manual.ltx}
\immediate\write18{mkindex hpitr-manual}
\immediate\write18{lualatex -interaction=nonstopmode hpitr-manual.ltx}

\immediate\write18{lualatex -interaction=nonstopmode example-single.tex}
\immediate\write18{lualatex -interaction=nonstopmode example-singlea.tex}
\immediate\write18{lualatex -interaction=nonstopmode example-proceedings.tex}

\ifToplevel{%
  \Msg{* \space\space done.}
  \Msg{*}
  \Msg{* You may install all the files now.}
  \Msg{*}
  \Msg{********************************************************************}
}
%    \end{macrocode}
% At docstrip, this concludes the run.
%    \begin{macrocode}
\csname endinput\endcsname
%    \end{macrocode}
% \iffalse meta-comment
%</dtx|ins>
% \fi^^A meta-comment
%^^A ==========================================================================
%^^A ==========================================================================
%^^A ==========================================================================
% \iffalse meta-comment
%<*class>
% \fi^^A meta-comment
%
% \subsection{The Tech Report Class}
%
%
% \subsubsection{Preparation}
%
%    \begin{macrocode}
%%%%%%%%%%%%%%%%%%%%%%%%%%%%
%% Basic tools
%%%%%%%%%%%%%%%%%%%%%%%%%%%%

\RequirePackage{etoolbox}
\RequirePackage{scrbase}[2013/12/19]


%%
%% Id and error
%%
\edef\TR@ClassName{\@currname}
\def\TR@warn{\ClassWarning{\TR@ClassName}}
\def\TR@info{\ClassInfo{\TR@ClassName}}
\def\TR@error{\ClassError{\TR@ClassName}}
\newcommand*\TR@fonterror{\TR@error{%
    Old font commands are not to be used}}

%%%%%%%%%%%%%%%%%%%%%%%%%%%%
%% Conditional statements
%%%%%%%%%%%%%%%%%%%%%%%%%%%%

\RequirePackage{ifthen,ifdraft,ifxetex,ifluatex}

%    \end{macrocode}
% 
% \subsubsection{Initialization}
%    \begin{macrocode}

%%%%%%%%%%%%%%%%%%%%%%%%%%%%
%% Options
%%%%%%%%%%%%%%%%%%%%%%%%%%%%


% reach into the gut of latex' kernel to push back global vars
\newcommand*\@pushback@classoption[1]{%
  \xdef\@classoptionslist{%
    \ifx\@classoptionslist\@empty\else\@classoptionslist,\fi
    #1}%
}


\newcommand*{\TR@NowButAtEndOfClass}{\AtEndOfClass}
\AtEndOfClass{\let\TR@NowButAtEndOfClass\@firstofone}

% from scrkbase

\DefineFamily{hpitr}
\def\TRFamily{\DefineFamilyMember{hpitr}}
\TRFamily%


\newcommand*{\TRKey}[1][.\@currname.\@currext]{%
  \DefineFamilyKey[#1]{hpitr}%
}
\newcommand*{\TRExecuteOptions}[1][.\@currname.\@currext]{%
  \FamilyExecuteOptions[#1]{hpitr}%
}
\newcommand*{\TRoptions}{\FamilyOptions{hpitr}}
\newcommand*{\AfterTRoptions}{}
\let\AfterTRoptions\AtEndOfFamilyOptions
\newcommand*{\TRoption}{\FamilyOption{hpitr}}

\newcommand*{\TRnewif}{\TRFamily\FamilyBoolKey{hpitr}}
\newcommand*{\TRsetif}{\FamilySetBool{hpitr}}
\newcommand*{\TRnewnumc}{\TRFamily\FamilyNumericalKey{hpitr}}
\newcommand*{\TRsetnumc}{\FamilySetNumerical{hpitr}}

\newcommand*{\TRnewifStd}[1]{%
  \newbool{TR@#1}
  \TRnewif{#1}{TR@#1}}

\newcommand*{\TR@curropt}{}
\newcommand*{\TRStdOption}[3][]{%
  \let\TR@curropt\CurrentOption
  \DeclareOption{#2}{
    #1%
    \TRExecuteOptions{#3}}
  \let\CurrentOption\TR@curropt
}

\providebool{@draft}
\newbool{TR@fulldraft}
\TRKey{draft}[true]{%
  \TRsetif{draft}{@draft}{#1}%
  \ifx\FamilyKeyState\FamilyKeyStateProcessed
    \boolfalse{TR@fulldraft}
  \else
    \ifstr{#1}{full}{%
      \booltrue{@draft}
      \booltrue{TR@fulldraft}
    }{}
  \fi
  \AfterTRoptions{
    \ifthenelse{\boolean{@draft}}{
      \@pushback@classoption{draft}
      \ifthenelse{\not\boolean{TR@fulldraft}}{
        \PassOptionsToPackage{draft=false}{scrlayer-scrpage}
        \PassOptionsToPackage{final}{listings}
%    \end{macrocode}
% \changes{v0.3a}{2014/08/08}{Require grffile for non-trivial file names}
%    For non-fulldraft to work with grffile, we unconditionally use the
%    graphics internal draft boolean.
%    \begin{macrocode}
        \AtBeginDocument{\Gin@draftfalse}
      }{}
    }{}
  }
}
\TRStdOption{final}{draft=false}
\TRStdOption{draft}{draft=true}


%% name of the base class
\providecommand*\TR@BaseClass{\@empty}

%% proceedings?
\newbool{TR@proceedings}
\newbool{TR@contribtools}
\newbool{TR@chapters}
\booltrue{TR@chapters}
%    \end{macrocode}
% \begin{option}{trtype}
%    \begin{macrocode}
\TRnewifStd{chapterbib}
\TRStdOption{chapterbib}{chapterbib=true}
%    \end{macrocode}
% \end{option}
%
% \begin{option}{trtype}
%    \begin{macrocode}
\TRKey{trtype}{
  \TRsetnumc{trtype}{@tempa}{%
    {singlearticle}{0},{article}{0},%
    {singlereport}{1},{report}{1},%
    {proceedings}{2},%
    {collection}{3}%
  }{#1}
  \ifcase \@tempa\relax
    \renewcommand*\TR@BaseClass{scrartcl}
    \boolfalse{TR@proceedings}
    \boolfalse{TR@chapters}
    \boolfalse{TR@contribtools}
  \or
    \renewcommand*\TR@BaseClass{scrreprt}
    \boolfalse{TR@proceedings}
    \booltrue{TR@chapters}
    \boolfalse{TR@contribtools}
  \or
    \renewcommand*\TR@BaseClass{scrbook}
    \booltrue{TR@proceedings}
    \booltrue{TR@contribtools}
    \booltrue{TR@chapters}
    \booltrue{TR@chapterbib}
  \or
    \renewcommand*\TR@BaseClass{scrbook}
    \boolfalse{TR@proceedings}
    \booltrue{TR@contribtools}
    \booltrue{TR@chapters}
  \fi
}
%    \end{macrocode}
% \end{option}
% \begin{option}{single}
% \begin{option}{singlereport}
% \begin{option}{singlearticle}
% \begin{option}{proceedings}
% \begin{option}{collection}
%    \begin{macrocode}
\TRStdOption{single}{trtype=singlereport}
\TRStdOption{singlereport}{trtype=singlereport}
\TRStdOption{singlearticle}{trtype=singlearticle}
\TRStdOption{proceedings}{trtype=proceedings}
\TRStdOption{collection}{trtype=collection}
%    \end{macrocode}
% \end{option}
% \end{option}
% \end{option}
% \end{option}
% \end{option}
%    \begin{macrocode}

% Deprecated
\DeclareOption{inproceedings}{
  \let\TR@curropt\CurrentOption
  \begingroup
  \TR@warn{%
    You've used obsolete option `inproceedings'.\MessageBreak
    Usage of this option indicates an old document that\MessageBreak
    used `polze.cls'.\MessageBreak
    \MessageBreak
    I switch to `trtype=singlearticle' for you.
  }
  \endgroup
  \TRExecuteOptions{trtype=singlearticle}%
  \let\CurrentOption\TR@curropt
}

\TRnewifStd{todotools}

%% execute the standard options
\TRExecuteOptions{
  trtype=singlereport,%
}

\FamilyProcessOptions{hpitr}\relax


%%%%%%%%%%%%%%%%%%%%%%%%%%%%
%% Load base class file
%%%%%%%%%%%%%%%%%%%%%%%%%%%%

\LoadClass{\TR@BaseClass}


%%
%% early stuff
%%
\ifthenelse{\boolean{xetex} \or \boolean{luatex}}{}{%
  \RequirePackage[utf8]{inputenc}
}


%    \end{macrocode}
%
% Set basic paper and typearea options
%    \begin{macrocode}
\KOMAoptions{%
  pagesize=auto,
  paper=a4,
  DIV=calc,
  twoside=semi,
}
\AtEndPreamble{\recalctypearea}
%    \end{macrocode}
%
%    \begin{macrocode}
\KOMAoptions{
  twocolumn=false,
  fontsize=12pt,
  footinclude=false,
  headinclude=true,
}
\ifthenelse{\boolean{TR@proceedings}}{
  \KOMAoption{open}{right}
}{}
%
% Fixes
%
%
\RequirePackage{fixltx2e}[2014/06/10]
\RequirePackage{scrhack}

\PassOptionsToPackage{log-declarations=false}{xparse}
\RequirePackage{expl3}[2011/09/05]
\RequirePackage{xparse}
% https://tex.stackexchange.com/questions/185713/fontspec-2-4-and-scale-matchlowercase-causes-error
\ExplSyntaxOn
\cs_if_free:NTF \fp_div:Nn
  {
    \cs_new_protected:Npn \fp_div:Nn #1 #2
      {
        \fp_set:Nn #1 { #1/#2 }
      }
  }{}
\ExplSyntaxOff


%    \end{macrocode}
%%%%%%%%%%%%%%%%%%%%%%%%%%%% 
%% Title.
%%
%% No actual title, only a titlepage page when in draft mode.
%%
%% However, we collect the information for the pdf and also for
%% convenience.
% \changes{v0.4}{2014/08/12}{Delay \cs{title}/\cs{author} pdf setting
% to latest point}
%%
%%%%%%%%%%%%%%%%%%%%%%%%%%%
%    \begin{macrocode}
\@ifpackagelater{scrbase}{2013/12/19}{%
  \KOMAoption{titlepage}{firstiscover}
}{
  \KOMAoption{titlepage}{yes}
}

\apptocmd{\title}{%
  \AfterPreamble{
    \begingroup
    \appto\pdfstringdefPreHook{\def\\{\space\ignorespaces}}
    \hypersetup{pdftitle={#1}}
    \endgroup}}
{}{%
  \TR@warn{Cannot patch \string\title to set pdftitle}}

\apptocmd{\author}{%
  \AfterPreamble{
    \begingroup
    \appto\pdfstringdefPreHook{\def\and{,\space\ignorespaces}}
    \hypersetup{pdfauthor={#1}}
    \endgroup}
  }
{}{%
  \TR@warn{Cannot patch \string\author to set pdfauthor}}

%    \end{macrocode}
%
% \begin{macro}{\email}
% \mbox{}
%\changes{v0.4}{2014/08/12}{Provide email command based on hyperref}
%    \begin{macrocode}
\providecommand*\email[1]{\href{mailto:#1}{\nolinkurl{#1}}\xspace}
%    \end{macrocode}
% \end{macro}
%%%%%%%%%%%%%%%%%%%%%%%%%%%% 
% Titlepage
%%%%%%%%%%%%%%%%%%%%%%%%%%%%
% \changes{v0.4}{2014/08/12}{If draft-titlepage, set its numbering in Roman
% (for pdftex)}
%    \begin{macrocode}

\ifthenelse{\boolean{@draft}}{
  \AtBeginDocument{
    \setkomafont{author}{\normalfont\normalsize}
    \setkomafont{date}{\normalfont\normalsize}
    \setkomafont{dedication}{\normalfont\normalsize}
    \setkomafont{publishers}{\normalfont\normalsize}
    \setkomafont{subtitle}{\normalfont\small}
    \setkomafont{title}{\normalfont\normalsize}
    \setkomafont{titlehead}{\normalfont\normalsize}
  }
  \pretocmd{\titlepage}{%
    \ifthenelse{\boolean{@mainmatter}}{\pagenumbering{Roman}}{}
    \let\center\flushright%
    \let\endcenter\endflushright%
    \let\huge\normalsize%
  }{}{%
    \TR@warn{Cannot patch \string\maketitle}}
  \apptocmd{\endtitlepage}{%
    \cleardoubleemptypage%
    \ifthenelse{\boolean{@mainmatter}}{%
      \pagenumbering{arabic}}{%
      \pagenumbering{roman}}
    \setcounter{page}{5}%
    \relax%
  }{}{%
    \TR@warn{Cannot patch \string\maketitle}}
}{%
  \def\maketitle{\setcounter{page}{5}\relax}
}


\ifthenelse{\boolean{TR@proceedings}\OR\boolean{TR@contribtools}}{
%    \end{macrocode}
%
%
%  Proceedings stuff.
% \begin{macro}{\event}
%% Title of the event
% \changes{v0.4}{2014/08/12}{Delay to latest point for hyperref}
%    \begin{macrocode}
  \newcommand*\event[1]{%
    \gdef\@eventtitle{#1}
    \AfterPreamble{\hypersetup{pdfsubject = {Technical Report: #1}}}
  }
  \def\@eventtitle{%
    \TR@warn{No \string\event\space given}
    \global\cslet{@eventtitle}\@empty}
%    \end{macrocode}
% \end{macro}
% \begin{macro}{\noevent}
% No event: clear it cleanly
% \changes{v0.4}{2014/08/12}{Provide \cs{noevent}}
%    \begin{macrocode}
  \newcommand*\noevent{\gdef\@eventtitle{}}
%% Paper number
  \newcommand{\papernumber}[1]{%
    \def\@papernumber{#1}
    \ifx\@papernumber\empty
    \else
      \renewcommand*{\thepage}{\@papernumber-\arabic{page}}
    \fi
  }
  \let\@papernumber\@empty

%    \end{macrocode}
% \end{macro}
% Collections
% \changes{v0.4}{2014/08/12}{Add facilities to make collections.}
%    \begin{macrocode}
  \RequirePackage{ragged2e}
  \AtEndPreamble{%
    \RequirePackage{biblatex}}
%    \end{macrocode}
% Automatically pick up \texttt{preamble.tex} and \texttt{content.tex} for each
% contribution.
%    \begin{macrocode}
  \providecommand\TR@contributions{}
  \newcommand*{\contributions}[1]{%
    \xdef\TR@contributions{#1}%
    \configureContributions%
  }
  \newcommand*{\TR@preamble}[1]{%
    \InputIfFileExists {#1/preamble}{}{}}

  \newcommand*{\configureContributions}{%
    \def\doit{}%
    \@for\next:=\TR@contributions\do{%
      \ifx\@empty\next\relax\else%
      \expandafter\appto\expandafter\doit\expandafter%
      {\expandafter\TR@preamble\expandafter{\next}}%
      \fi%
    }%
    \doit{}%
  }
  \newcommand*{\includeContributionContents}{%
    \def\doit{}%
    \@for\next:=\TR@contributions\do{%
      \ifx\@empty\next\else%
      \expandafter\appto\expandafter\doit\expandafter%
      {\expandafter\TR@includedir\expandafter{\next}}%
      \fi%
    }%
    \doit{}%
  }
%    \end{macrocode}
% Include \texttt{content.tex}, make it appear local and pick up bibliography.
%    \begin{macrocode}
  \newcommand*{\TR@includedir}[1]{%
    \global\let\TR@saved@include\include
    \global\let\TR@saved@input\input
    \gdef\include##1{\TR@saved@include{#1/##1}}
    \gdef\input##1{\TR@saved@input{#1/##1}}
    \graphicspath{{#1/}}
    \TR@includedir@{#1}
    \global\let\include\TR@saved@include%
    \global\let\input\TR@saved@input%
  }
  \ifthenelse{\boolean{TR@chapterbib}}{%
    \newcommand*{\TR@includedir@}[1]{%
      \begin{refsection}[#1/references.bib]
        \ifthenelse{\boolean{TR@proceedings}}{%
          \cleardoubleemptypage}{\clearpage}%
        \TR@saved@input{#1/content}%
        \printbibliography[%
          heading=subbibliography, %
          title=\refname, %
        ]%
      \end{refsection}}
  }{%
    \newcommand*{\TR@includedir@}[1]{%
      \ifthenelse{\boolean{TR@proceedings}}{\cleardoubleemptypage}{\clearpage}%
      \TR@saved@input{#1/content}%
    }
  }

%    \end{macrocode}
%
% Support for multiple seperate work titles, portions from llncs.cls
%    \begin{macrocode}
  \def\clearheadinfo{\gdef\@author{No Author Given}%
    \gdef\@title{No Title Given}%
    \gdef\@subtitle{}%
    \gdef\@institute{No Institute Given}%
    \gdef\@thanks{}%
    \global\let\@titlerunning\@empty% only non-proceedings
  }
  \newkomafont{abstract}{\normalfont\small}

  \ifthenelse{\boolean{TR@proceedings}}{%
    \newcounter{@inst}
    \newcounter{@auth}
    \newcounter{auco}
    \newbox\headrun
    \newtoks\authorrunning
    \newtoks\tocauthor
    \newtoks\titlerunning
    \newtoks\toctitle

    \def\institute#1{\gdef\@institute{#1}}

    \def\institutename{\par\begingroup
      \parskip=\z@
      \parindent=\z@
      \setcounter{@inst}{1}%
      \def\and{\par\stepcounter{@inst}%
        \noindent$^{\the@inst}$\enspace\ignorespaces}%
      \setbox0=\vbox{\def\thanks##1{}\@institute}%
      \ifnum\c@@inst=1\relax
        \gdef\fnnstart{0}%
      \else
        \xdef\fnnstart{\c@@inst}%
        \setcounter{@inst}{1}%
        \noindent$^{\the@inst}$\enspace
      \fi
      \ignorespaces
      \@institute\par
      \endgroup}

    \def\inst#1{\unskip$^{#1}$}
    \def\fnmsep{\unskip$^,$}

%    \end{macrocode}
% ToC-formatters for title and author, also llncs style (proceedings)
%    \begin{macrocode}
    \def\l@title#1#2{%
      \addpenalty{-\@highpenalty}
      \addvspace\baselineskip
      \@tempdima \z@
      \begingroup
        \parindent \z@ \rightskip \@tocrmarg
        \advance\rightskip by 0pt plus 2cm
        \parfillskip -\rightskip \pretolerance=10000
        \leavevmode \advance\leftskip\@tempdima \hskip -\leftskip
        \begingroup%
          \setlength{\RaggedRightRightskip}{0pt plus 1fil}%
          \RaggedRight%
          #1\nobreak
          \leaders\hbox{$\m@th \mkern \@dotsep mu.\mkern
            \@dotsep mu$}\hfill
          \nobreak\hbox to\@pnumwidth{\hss #2}\par
        \endgroup
        % \penalty\@highpenalty \endgroup
        \penalty10000\endgroup}

      \def\l@author#1#2{%\addpenalty{\@highpenalty}
        \addpenalty{10000}
        \@tempdima=15\p@ %\z@
        \begingroup%
          \parindent \z@ \rightskip \@tocrmarg%
          \advance\rightskip by 0pt plus 2cm%
          \pretolerance=10000
          \leavevmode \advance\leftskip\@tempdima %\hskip -\leftskip
          {\itshape #1\par}
          % \penalty\@highpenalty
          \penalty0\relax
          \endgroup}

    %%% to avoid hyperref warnings
    \providecommand*{\toclevel@author}{999}
    %%% to make title-entry parent of section-entries
    \providecommand*{\toclevel@title}{0}
%%% /if proceedings
  }{}
%    \end{macrocode}
% Patch maketitle to allow one global and multiple local titles.
%    \begin{macrocode}
  \newbool{TR@firsttitleseen}
  \boolfalse{TR@firsttitleseen}

  \let\TR@firsttitle@maketitle\maketitle
  \renewcommand*\maketitle{
    \ifthenelse{\boolean{TR@firsttitleseen}}{%
      \TR@contrib@maketitle%
      }{
        \TR@firsttitle@maketitle%
        \clearheadinfo%
        \booltrue{TR@firsttitleseen}}}
    % we switch to title seen also when people do not use \maketitle
    \apptocmd{\tableofcontents}{\booltrue{TR@firsttitleseen}}{}{%
      \TR@warn{Cannot patch \string\tableofcontents}}

  \let\TR@firsttitle@author\author
  \renewcommand*\author{
    \ifthenelse{\boolean{TR@firsttitleseen}}%
    {\TR@contrib@author}%
    {\TR@firsttitle@author}}

  \AtEndOfClass{
    \let\TR@firsttitle@keywords\keywords
    \renewcommand*\keywords{%
      \ifthenelse{\boolean{TR@firsttitleseen}}%
      {\TR@contrib@keywords}%
      {\TR@firsttitle@keywords}}
  }

  \providecaptionname{english}{\andname}{and}
  \providecaptionname{english}{\lastandname}{\unskip, and}
  \providecaptionname{german}{\andname}{und}
  \providecaptionname{german}{\lastandname}{und}
  \providecaptionname{ngerman}{\andname}{und}
  \providecaptionname{ngerman}{\lastandname}{und}

  \def\TR@authcount#1{\setcounter{auco}{#1}\setcounter{@auth}{1}}
  \pretocmd{\tableofcontents}{%
    \let\authcount\TR@authcount%
    \def\lastand{\ifnum\value{auco}=2\relax
        \unskip{} \andname\
      \else
        \unskip \lastandname\
      \fi}%
    \def\and{\stepcounter{@auth}\relax
      \ifnum\value{@auth}=\value{auco}%
        \lastand
      \else
        \unskip,
      \fi}%
  }{}{%
    \TR@warn{Cannot patch \string\tableofcontents}}

  \ifthenelse{\boolean{TR@proceedings}}{%
    \newcommand*\TR@contrib@maketitle{\newpage
      \phantomsection
      \refstepcounter{chapter}%
      \stepcounter{section}%
      \setcounter{section}{0}%
      \setcounter{subsection}{0}%
      \setcounter{figure}{0}
      \setcounter{table}{0}
      \setcounter{equation}{0}
      \setcounter{footnote}{0}%
      \begingroup
      \parindent=\z@
      \renewcommand\thefootnote{\@fnsymbol\c@footnote}%
      \newpage
      \global\@topnum\z@   % Prevents figures from going at top of page.
      \TR@contrib@@maketitle
      \thispagestyle{\chapterpagestyle}%
      \@thanks
      \def\\{\unskip\ \ignorespaces}\def\inst##1{\unskip{}}%
      \def\thanks##1{\unskip{}}\def\fnmsep{\unskip}%
      \if!\the\toctitle!\addcontentsline{toc}{title}{\@title}\else
        \addcontentsline{toc}{title}{\the\toctitle}\fi
      \if!\the\titlerunning!\else
        \edef\@title{\the\titlerunning}%
      \fi
      \ifx\@author\empty\relax%
        \xdef\@runninghead{\ignorespaces\@title}
      \else
        \if!\the\tocauthor!\relax
          {\def\and{\noexpand\protect\noexpand\and}%
            \protected@xdef\toc@uthor{\@author}}%
        \else
          \def\\{\noexpand\protect\noexpand\newline}%
          \protected@xdef\scratch{\the\tocauthor}%
          \protected@xdef\toc@uthor{\scratch}%
        \fi
        \addtocontents{toc}{\noexpand\protect\noexpand\authcount{\the\c@auco}}%
        \addcontentsline{toc}{author}{\toc@uthor}%

        \if!\the\authorrunning!
          \value{@inst}=\value{@auth}%
          \setcounter{@auth}{1}%
        \else
          \edef\@author{\the\authorrunning}%
        \fi

        \xdef\@runninghead{\unskip\ignorespaces\@author: \@title}
      \fi

      \global\setbox\headrun=\hbox{\usekomafont{pagehead}\@runninghead}%
      \ifdim\wd\headrun>\hsize
        \TR@warn{Title and author names too long for running head. Please supply
          a shorter form with \string\titlerunning\space and
          \string\authorrunning\space prior to \string\maketitle}%
        \xdef\@runninghead{Authors/Title Suppressed Due to Excessive Length}%
      \fi
      \addchapmark{\@runninghead}%
      \endgroup
      \setcounter{footnote}{\fnnstart}%
      \clearheadinfo%
      \relax%
      \ignorespaces}

    \newcommand*\TR@contrib@@maketitle{\newpage
      \def\lastand{\ifnum\value{@inst}=2\relax
          \unskip{} \andname\
         \else
           \unskip \lastandname\
         \fi}%
       \def\and{\stepcounter{@auth}\relax
         \ifnum\value{@auth}=\value{@inst}%
           \lastand
         \else
           \unskip,
         \fi}%
       \begin{center}%
         \let\newline\\
         {\Large \bfseries\boldmath
           \pretolerance=10000
           \@title \par}\vskip .8cm
         \if!\@subtitle!\else {\large \bfseries\boldmath
           \vskip -.65cm
           \pretolerance=10000
           \@subtitle \par}\vskip .8cm\fi
         \setbox0=\vbox{\setcounter{@auth}{1}\def\and{\stepcounter{@auth}}%
           \def\thanks##1{}\@author}%
         \global\value{@inst}=\value{@auth}%
         \global\value{auco}=\value{@auth}%
         \setcounter{@auth}{1}%
         {\lineskip .5em
           \noindent\ignorespaces
           \@author\vskip.35cm}
         {\small\institutename}
       \end{center}%
     }

     \newcommand*\TR@contrib@author[1]{\gdef\@author{#1}}
     \newcommand*{\TR@contrib@keywords}[1]{%
       \par\addvspace\baselineskip%
       \noindent\keywordname\enspace\ignorespaces#1}%

  \renewcommand*\thesection{\@arabic\c@section}
  \renewcommand*\thefigure{\@arabic\c@figure}
  \renewcommand*\thetable{\@arabic\c@table}
  \renewcommand*\theequation{\@arabic\c@equation}
  \@addtoreset{equation}{chapter}
  \BeforePackage{hyperref}{%
    \newcommand*\theHequation{\theHsection.\arabic{equation}}%
    \newcommand*\theHchapter    {\arabic{chapter}}%
    \newcommand*\theHfigure     {\theHchapter.\arabic{figure}}%
    \newcommand*\theHtable      {\theHchapter.\arabic{table}}%
    \newcommand*\theHsection    {\theHchapter.\arabic{section}}%
    \newcommand*\theHsubsection   {\theHsection.\arabic{subsection}}
    \newcommand*\theHsubsubsection{\theHsubsection.\arabic{subsubsection}}
    \newcommand*\theHparagraph    {\theHsubsubsection.\arabic{paragraph}}
    \newcommand*\theHsubparagraph {\theHparagraph.\arabic{subparagraph}}
    \newcommand*\theHtheorem      {\theHsection.\arabic{theorem}}
    \newcommand*\theHthm          {\theHsection.\arabic{thm}}
  }


  }{%%%% collection
    \let\TR@contrib@author\@gobble
    \let\TR@contrib@keywords\@gobble
    \let\institute\@gobble
    \let\authorrunning\@gobble
    \let\toctitle\@gobble
    \let\tocauthor\@gobble

    \let\@titlerunning\@empty
    \setkomafont{abstract}{\footnotesize\itshape}
    \newcommand*\titlerunning[1]{\gdef\@titlerunning{#1}}
    \newcommand*\TR@contrib@maketitle{%
      \ifcsempty{@titlerunning}{%
        \expandafter\chapter\expandafter{\@title}%
      }{%
        \expandafter\chapter\expandafter[\expandafter\@titlerunning\expandafter]\expandafter{\@title}%
      }
      \edef\@templ{lbl:chaper-\thechapter}%
      \expandafter\label\expandafter{\@templ}%
      \clearheadinfo%
    }
  }

  \AfterPackage{listings}{
    \lstset{numberbychapter=true}}
  \ifthenelse{\boolean{TR@proceedings}}{%
    \AfterPackage{listings}{
      \AtBeginDocument{
        \def\thelstlisting{\@arabic\c@lstlisting}%
        \def\theHlstnumber{\ifx\lst@@caption\@empty \lst@neglisting
                                                 \else \theHlstlisting \fi%
                                                 .\thelstnumber}}}
  }{}

%    \end{macrocode}
% as for title, as for abstract.
%    \begin{macrocode}
    \newenvironment{abstract}{%
      \ifthenelse{\boolean{TR@firsttitleseen}}{
        \list{}{%\advance\topsep by0.35cm\relax
          \setlength{\topsep}{0pt}%
          \setlength{\leftmargin}{2em}%
          \setlength{\rightmargin}{\leftmargin}%
          \setlength{\listparindent}{\parindent}%
          \setlength{\itemindent}{\parindent}%
          \setlength{\parsep}{\parskip}%
          \usekomafont{abstract}
          }\item[]\noindent\ignorespaces%
      }{%
        \TR@error{Proceedings should not have \MessageBreak
          abstracts. Please remove it.}}
    }%
    {\ifthenelse{\boolean{TR@firsttitleseen}}{\endlist}{}}

  %for compat with swathesis
  \let\supervisors\@gobble
  \let\statements\relax
}{}
%    \end{macrocode}
% \begin{macro}{\keywords}
% \begin{macro}{\test}
% \changes{v0.6}{2014/08/27}{Provide \cs{keywords} command.}
% We provide a \cs{keyword} macro to record keywords for the printed PDF.
% Should there already be one, we patch it like \cs{title} and \cs{author}
%    \begin{macrocode}
\newcommand*\keywords[1]{
  \AfterPreamble{
    \begingroup
    \appto\pdfstringdefPreHook{\def\\{\space\ignorespaces}}
    \hypersetup{pdfkeywords={#1}}
    \endgroup}
}
%    \end{macrocode}
% \end{macro}
% \end{macro}
%%%%%%%%%%%%%%%%%%%%%%%%%%%% 
%% Language
%%%%%%%%%%%%%%%%%%%%%%%%%%%%
%    \begin{macrocode}
\RequirePackage{hyphsubst}
\HyphSubstLet{ngerman}{ngerman-x-latest}
\RequirePackage{babel}
\newcommand*\TR@lang{}\let\TR@lang\@empty

%    \end{macrocode}
% \changes{v0.4}{2014/08/12}{Silence babel+hyperref+shorthands}
%    \begin{macrocode}
\AfterPackage{hyperref}{
  % https://tex.stackexchange.com/questions/193025/babel-hyperref-redefining-shorthand-several-times-per-page
  \patchcmd{\pdfstringdef}
    {\csname HyPsd@babel@}
    {\let\bbl@info\@gobble\csname HyPsd@babel@}
    {}{}
}


%    \end{macrocode}
% extract ISO 639-2 code from babel language name
% \DoNotIndex{\@l,\@L}
%    \begin{macrocode}
{
  \def\@l#1{\equal{\languagename}{#1}}
  \def\@L#1{\gdef\TR@lang{#1}}
\ifthenelse{\@l{afrikaans}}{\@L{af}}{}
\ifthenelse{\@l{bahasa}\OR\@l{indonesian}\OR\@l{indon}\OR%
  \@l{bahasai}\OR\@l{bahasam}\OR\@l{malay}\OR\@l{meyalu}}{\@L{id}}{}
\ifthenelse{\@l{basque}}{\@L{eu}}{}
\ifthenelse{\@l{breton}}{\@L{br}}{}
\ifthenelse{\@l{bulgarian}}{\@L{bg}}{}
\ifthenelse{\@l{catalan}}{\@L{ca}}{}
\ifthenelse{\@l{croatian}}{\@L{hr}}{}
\ifthenelse{\@l{czech}}{\@L{cs}}{}
\ifthenelse{\@l{danish}}{\@L{da}}{}
\ifthenelse{\@l{dutch}}{\@L{nl}}{}
\ifthenelse{\@l{english}\OR\@l{USenglish}\OR\@l{american}\OR%
  \@l{UKenglish}\OR\@l{british}\OR\@l{canadian}\OR%
  \@l{australian}\OR\@l{newzealand}}{\@L{en}}{}
\ifthenelse{\@l{esperanto}}{\@L{eo}}{}
\ifthenelse{\@l{estonian}}{\@L{et}}{}
\ifthenelse{\@l{finnish}}{\@L{fi}}{}
\ifthenelse{%
  \@l{french}\OR\@l{francais}\OR\@l{canadien}\OR\@l{acadian}}{\@L{fr}}{}
\ifthenelse{\@l{galician}}{\@L{gl}}{}
\ifthenelse{\@l{german}\OR\@l{germanb}\OR\@l{ngerman}\OR%
  \@l{ngermanb}\OR\@l{austrian}\OR\@l{naustrian}}{\@L{de}}{}
\ifthenelse{\@l{greek}\OR\@l{polutonikogreek}}{\@L{el}}{}
\ifthenelse{\@l{hebrew}}{\@L{he}}{}
\ifthenelse{\@l{magyar}\OR\@l{hungarian}}{\@L{hu}}{}
\ifthenelse{\@l{icelandic}}{\@L{is}}{}
\ifthenelse{\@l{interlingua}}{\@L{ia}}{}
\ifthenelse{\@l{irish}}{\@L{ga}}{}
\ifthenelse{\@l{italian}}{\@L{it}}{}
\ifthenelse{\@l{latin}}{\@L{la}}{}
\ifthenelse{\@l{samin}}{\@L{se}}{}
\ifthenelse{\@l{norsk}}{\@L{no}}{}
\ifthenelse{\@l{nynorsk}}{\@L{nn}}{}
\ifthenelse{\@l{polish}}{\@L{pl}}{}
\ifthenelse{%
  \@l{portuges}\OR\@l{portuguese}\OR\@l{brazilian}\OR\@l{brazil}}{\@L{pt}}{}
\ifthenelse{\@l{romanian}}{\@L{ro}}{}
\ifthenelse{\@l{russian}}{\@L{ru}}{}
\ifthenelse{\@l{scottish}}{\@L{gd}}{}
}
%    \end{macrocode}
% \subsubsection{Fonts}
% Default font: Serif:  Palatino via \TeX Gyre Pagella\\
% Code Font: Monospace: Source Code Pro (Adobe)\\
% Other emphasis: Sans: Source Sans Pro (Adobe)\\
%    \begin{macrocode}
\ifthenelse{\boolean{luatex}}{%
  \RequirePackage[final]{microtype}[2013/05/23]
}{
  \RequirePackage[final,babel=true]{microtype}[2013/05/23]
}
%
%    \end{macrocode}
%  engine-dependent, eg, encoding, loading.
%    \begin{macrocode}
%
\RequirePackage[T1]{fontenc}
\ifthenelse{\boolean{luatex} \or \boolean{xetex}}{%
  \RequirePackage{fontspec}[2014/06/21]
}{}

% math font: euler
\RequirePackage[euler-digits,T1]{eulervm}
\let\mathbf\mathbold
\ifthenelse{\boolean{luatex} \or \boolean{xetex}}{%
%    \end{macrocode}
% \changes{v0.4}{2014/08/12}{Load TeX Gyre Pagella before Adobe Source for
% MatchLowercase reference.}
%    \begin{macrocode}
  \setmainfont{texgyrepagella}[ExternalLocation,%
      Numbers={Proportional,OldStyle},%
      UprightFont= *-regular,%
      BoldFont=*-bold,%
      ItalicFont=*-italic,%
      SlantedFont=*-italic,%
      BoldItalicFont=*-bolditalic,%
      BoldSlantedFont=*-bolditalic,%
      Ligatures={Common,TeX},%
      ]
  \RequirePackage[scale=MatchLowercase,semibold]{sourcecodepro}
  \RequirePackage[scale=MatchLowercase,semibold,osf]{sourcesanspro}
}{
  \RequirePackage[scale=.9,semibold]{sourcecodepro}
  \RequirePackage[scale=.9,semibold,osf]{sourcesanspro}
  \RequirePackage[osf,sc]{mathpazo}
}
\linespread{1.05} % a bit more for Palatino


%    \end{macrocode}
% Akin to LNCS, let footnotes be a bit larger
%    \begin{macrocode}
\let\footnotesize\small

%    \end{macrocode}
% \changes{v0.4}{2014/08/12}{Disable tracking=smallcaps in luatex, also. Is broken.}
%    \begin{macrocode}
\ifthenelse{\boolean{xetex}\OR\boolean{luatex}}{
  \microtypesetup{stretch=9,shrink=15,step=3}
}{
  \microtypesetup{stretch=9,shrink=15,step=3,tracking=smallcaps,letterspace=75}
}

\addtokomafont{disposition}{\rmfamily}
\setkomafont{title}{\rmfamily\mdseries}
\addtokomafont{descriptionlabel}{\rmfamily}
\renewcommand*\labelitemii{\normalfont\textendash}
\setfootnoterule{4cc}


%%%%%%%%%%%%%%%%%%%%%%%%%%%%
%% Text companion fonts
%%%%%%%%%%%%%%%%%%%%%%%%%%%%

\RequirePackage{textcomp}
\RequirePackage{mathcomp}
\RequirePackage{relsize}

%%%
%%% Details
%%%
\KOMAoptions{
  headings=big
}
%%%%%%%%%%%%%%%%%%%%%%%%%%%%
%% Page breaks
%%%%%%%%%%%%%%%%%%%%%%%%%%%%

%% no footnote breaks
\interfootnotelinepenalty = 1000

%% avoid widows (Hurenkinder) and orphans (Schusterjungen)
\clubpenalty = \@M
\widowpenalty = \@M

\setlength{\textfloatsep}{2\baselineskip}
\setlength{\floatsep}{\baselineskip}
\setlength{\intextsep}{2\baselineskip}

\AtEndPreamble{
  \frenchspacing
  \raggedbottom
}


%%%%%%%%%%%%%%%%%%%%%%%%%%%%
%% Headers and footers
%%%%%%%%%%%%%%%%%%%%%%%%%%%%

\KOMAoptions{
  headsepline=false,
  footsepline=false,
}
\RequirePackage{scrlayer-scrpage}
\pagestyle{scrheadings}

\ifthenelse{\boolean{TR@chapters}}{
  \renewcommand*\chapterpagestyle{plain.scrheadings}
  \automark[chapter]{chapter}
  \automark*[section]{}
}{
  \lehead{\@title}
  \automark*[section]{}
}

%    \end{macrocode}
% \changes{v0.4}{2014/08/12}{Fix pagenumber}
%    \begin{macrocode}
\lefoot[]{}\lofoot[]{}
\refoot[]{}\rofoot[]{}
\cefoot[\pagemark]{\pagemark}
\cofoot[\pagemark]{\pagemark}

%%  \ifthenelse{\boolean{TR@proceedings}\OR\boolean{TR@contribtools}}{
%%   \lefoot[\@eventtitle]{\@eventtitle}
%%   \lofoot[\@eventtitle]{\@eventtitle}
%% }{}

 %\setkomafont{pagefoot}{}
\setkomafont{pagehead}{\itshape}
\setkomafont{pageheadfoot}{\normalcolor}
\setkomafont{pagenumber}{\normalfont}
%
%
% Sectioning
%

\ifthenelse{\boolean{TR@chapters}}{
  \renewcommand*{\chapterheadstartvskip}{\vspace*{3\baselineskip}}
  \renewcommand*{\chapterheadendvskip}{\vspace*{2\baselineskip}}
}{}
% for sections, not so easy.
\def\TR@patchw{\TR@warn{Cannot patch sectioning command.\MessageBreak This is
    not severe.}}
\patchcmd{\section}{-3.5ex \@plus -1ex \@minus -.2ex}{-2\baselineskip}{}{\TR@patchw}
\patchcmd{\section}{2.3ex \@plus.2ex}{\baselineskip}{}{\TR@patchw}
\patchcmd{\subsection}{-3.25ex\@plus -1ex \@minus -.2ex}{-\baselineskip}{}{\TR@patchw}
\patchcmd{\subsection}{1.5ex \@plus .2ex}{\baselineskip}{}{\TR@patchw}
\patchcmd{\subsubsection}{-3.25ex\@plus -1ex \@minus -.2ex}{-\baselineskip}{}{\TR@patchw}
\patchcmd{\subsubsection}{1.5ex \@plus .2ex}{1sp \@plus 1sp \@minus 1sp}{}{\TR@patchw}
\patchcmd{\paragraph}{3.25ex \@plus1ex \@minus .2ex}{\baselineskip}{}{\TR@patchw}
\patchcmd{\subparagraph}{3.25ex \@plus1ex \@minus .2ex}{\baselineskip}{}{\TR@patchw}



%    \end{macrocode}
%%%%%%%%%%%%%%%%%%%%%%%%%%%%
%% PDF links and bookmarks
%%%%%%%%%%%%%%%%%%%%%%%%%%%%
% \changes{v0.4}{2014/08/12}{Delay hyperreft to the latest possible point.}
%    \begin{macrocode}


\PassOptionsToPackage{hyphens}{url}
\PassOptionsToPackage{%
  final,%
  unicode=true,%
  plainpages=false,%
  pdfpagelabels=true,%
}{hyperref}
\AtEndPreamble{%
  \RequirePackage{hyperref}%
  \RequirePackage{hyperxmp}
}
\AtBeginDocument{%
  \hypersetup{%
    breaklinks=true,
    %pdfborder = 0 0 0,
    bookmarksnumbered = true,
    pdfsubject = {Technical Report},
    pdfkeywords = {},
    pdfcreator = {HPI},
    pdfcopyright = {Copyright (c) \the\year, HPI},
    pdfmetalang = {en},
    pdfproducer = {\TR@ClassName - Technical Reports at the HPI},
    pdflang = {\TR@lang},
    pdfdisplaydoctitle = false,
    pdfpagemode = UseOutlines,
  }
}
%%
%%
%% Table of contents
%%
%% no protrusion there
%% starts on own page.

\RequirePackage{tocbasic}
\setcounter{tocdepth}{1}
\tocbasicautomode
\setuptoc{toc}{noprotrusion}

\AfterPackage{listings}{
  \setuptoc{lol}{noprotrusion}
}

\ifthenelse{\boolean{TR@chapters}}{
  \renewcommand*\raggedchapterentry{\raggedright}
}

%%%%%%%%%%%%%%%%%%%%%%%%%%%%
%% Abstract
%%%%%%%%%%%%%%%%%%%%%%%%%%%%

\ifthenelse{\boolean{TR@proceedings}\OR\boolean{TR@contribtools}}{}{
  \renewenvironment{abstract}{%
    \thispagestyle{plain.scrheadings}
    \null\vfil\leavevmode%
    \noindent%
    \ignorespaces
  }{%
    \par\vfil%
    \cleardoublepage%
  }
}

%    \end{macrocode}
% \subsubsection{Bibliography}
% \changes{v0.4}{2014/08/12}{Provide common biblatex ground}
%    \begin{macrocode}
\PassOptionsToPackage{style=numeric-comp,firstinits=true}{biblatex}
\AfterPackage{biblatex}{
  \setcounter{biburllcpenalty}{7000}
  \setcounter{biburlucpenalty}{8000}
  \ExecuteBibliographyOptions{
    url=false,%
    urldate=iso8601,%
    abbreviate=false,%
    maxnames=20,%
  }
}
\AtEndPreamble{
  \@ifpackageloaded{biblatex}{}{
% normal bibtex
    \bibliographystyle{plain}
  }
}

%%%%
%%
%%%
%    \end{macrocode}
% Use references name instead of bibliography throughout
% \changes{v0.4}{2014/08/12}{More robust variand of refname forcing}
%    \begin{macrocode}
\ifthenelse{\boolean{TR@chapters}}{%
  \global\let\@auto@bibname\refname
  \global\let\bibname\refname
  \defcaptionname{ngerman}{\bibname}{\refname}
  \defcaptionname{german}{\bibname}{\refname}
  \defcaptionname{english}{\bibname}{\refname}
}{}

%%%%%%%%%%%%%%%%%%%%%%%%%%%%
%% Math mode
%%%%%%%%%%%%%%%%%%%%%%%%%%%%

\RequirePackage{amsmath}
\RequirePackage{amssymb}


%%%%%%%%%%%%%%%%%%%%%%%%%%%%
%% Tables
%%%%%%%%%%%%%%%%%%%%%%%%%%%%

\RequirePackage{tabularx}
\RequirePackage{longtable}
\RequirePackage{booktabs}

%%%%%%%%%%%%%%%%%%%%%%%%%%%%
%% Graphics and Floats
%%%%%%%%%%%%%%%%%%%%%%%%%%%%

\RequirePackage{graphicx}
%    \end{macrocode}
% \changes{v0.4}{2014/08/12}{Require grffile for non-trivial file names}
%    \begin{macrocode}
\RequirePackage{grffile}

\AtEndPreamble{
  \@ifpackageloaded{float}{}{% mock float
    \providecommand*\floatname[2]{\@namedef{fname@#1}{#2}}
    \providecommand*\floatplacement[2]{\@namedef{fps@#1}{#2}}
    \floatplacement{figure}{htbp}
    \floatplacement{table}{htp}
  }
}

\AfterPackage{float}{
% disable [H]
  \def\@float@HH#1{%
    \TR@error{The [H] specifier for floats is forbidden for\MessageBreak
      the hpitr class.}}
  \floatstyle{komabelow}
}

\setcapindent{1em}
\addtokomafont{captionlabel}{\bfseries}
\PassOptionsToPackage{ruled}{algorithm2e}

\providecommand*\TR@listingfont{}
\ifthenelse{\boolean{xetex}}{%
  \newcommand*\TRlistingfont[1]{\gdef\TR@listingfont{#1\addfontfeature{Mapping=}}}%
}{%
  \ifthenelse{\boolean{luatex}}{%
    \newcommand*\TRlistingfont[1]{\gdef\TR@listingfont{#1\addfontfeature{RawFeature={-tlig;-trep}}}}%
  }{
    \newcommand*\TRlistingfont[1]{\gdef\TR@listingfont{#1}}}
}{}
\TRlistingfont{\ttfamily\small}

\AfterPackage{verbatim}{
  \renewcommand*\verbatim@font{\TR@listingfont}
}
\AfterPackage{listings}{
  \AtEndPreamble{
    \lstset{%
      basicstyle=\TR@listingfont,
      floatplacement=htbp,
      captionpos=t,
      abovecaptionskip=0pt,
      belowcaptionskip=0pt,
      upquote=true,
      showstringspaces=false,
      inputencoding=utf8,
    }
  }
}
%%%%%%%%%%%%%%%%%%%%%%%%%%%%
%% Color
%%%%%%%%%%%%%%%%%%%%%%%%%%%%

\RequirePackage{color}
\RequirePackage[table,svgnames,dvipsnames]{xcolor}

%%%%%%%%%%%%%%%%%%%%%%%%%%%%
%% include PDF Documents
%% with more than 1 page
%%%%%%%%%%%%%%%%%%%%%%%%%%%%

\ifthenelse{\boolean{TR@proceedings}}{
  \RequirePackage{pdfpages}
  \ifthenelse{\boolean{xetex}}{}{\pdfinclusioncopyfonts = 1}
  \includepdfset{pagecommand={\thispagestyle{plain.scrheadings}}}
}{}



% %%%%%%%%%%%%%%%%%%%%%%%%%%%%
% %% Comments and Todo help
% %%%%%%%%%%%%%%%%%%%%%%%%%%%%

\ifthenelse{\boolean{TR@todotools}}{%
  \RequirePackage[obeyDraft,colorinlistoftodos]{todonotes}
  \newcommand*\todosec{\par\noindent\todo[inline]}
  \newcommand*\secmissing{\par\noindent\todo[color=red,inline,size=\Large]}
  \newcommand\todolist[2]{%
    \par\noindent%
    \todo[inline,color={red!100!green!50},caption={#1}]{%
      \begin{minipage}{\linewidth}%
        \begin{itemize}
          #2%
        \end{itemize}
      \end{minipage}
    }%
  }
  \newcommand\todoauthor[2][\empty]{
    \expandafter\newcommand\csname #2\endcsname[2][\empty]{%
      \todo[color=#1,##1]{#2: ##2}}}
}


%%%%%%%%%%%%%%%%%%%%%%%%%%%%%%%%%%%%%%%%%%%%%%%%%%%%%%%%%%%%%%%%
%%
%%  Draft Stuff
%%
%%%%%%%%%%%%%%%%%%%%%%%%%%%%%%%%%%%%%%%%%%%%%%%%%%%%%%%%%%%%%%%%

\ifthenelse{\boolean{@draft}}{%
  \RequirePackage{blindtext}
  \errorcontextlines=999
  \RequirePackage{eso-pic}
  \newkomafont{draftline}{\sffamily}
  \newsavebox{\TR@draftPageLine}
  \AddToShipoutPicture{%
    \AtPageUpperLeft{%
      \raisebox{-\height}[\height][0pt]{\usebox{\TR@draftPageLine}}}%
    \AtPageLowerLeft{%
      \raisebox{\depth}[\height][0pt]{\usebox{\TR@draftPageLine}}}%
  }
  \AddToShipoutPicture{
    \begingroup
    \setlength{\@tempdima}{.5pt}%
    \setlength{\@tempdimb}{\dimexpr\paperwidth-1.75pt\relax}%
    \setlength{\@tempdimc}{\dimexpr\paperheight-1.5pt\relax}%
    \thicklines%
    \put(\LenToUnit{\@tempdima},\LenToUnit{\@tempdima}){%
      \framebox(\LenToUnit{\@tempdimb},\LenToUnit{\@tempdimc}){}}%
    \endgroup
  }
  \AtBeginDocument{
    \hypersetup{
      colorlinks = true,
      linkcolor=MidnightBlue,%
      citecolor=MidnightBlue,%
      urlcolor=MidnightBlue,%
    }
    \def\TR@draftInfo{%
      {\usekomafont{draftline}
      Draft Draft Draft%
      \hspace*{4cm}\today\hspace*{4cm}%
      Draft Draft Draft%
    }}
    \sbox{\TR@draftPageLine}{%
      \colorbox{black!10}{%
        % enlarge box vertically by 2/3 lines
        \raisebox{0pt}%
        [\dimexpr .33\baselineskip + \height]%
        [\dimexpr .33\baselineskip + \depth]{%
          \makebox[\paperwidth]{\color{black!50}\TR@draftInfo}}}}
  }
%    \end{macrocode}
% \changes{v0.3a}{2014/08/08}{Fix listoftodos, only to appear with todotools}
%    \begin{macrocode}
  \ifthenelse{\boolean{TR@todotools}}{
    \AtEndDocument{\listoftodos}}{}
}{
  \let\blindtext\relax
  \let\Blindtext\relax
  \let\blinddocument\relax
  \let\Blinddocument\relax
}

%%%%%%%%%%%%%%%%%%%%%%%%%%%%
%% Show the page layout
%%%%%%%%%%%%%%%%%%%%%%%%%%%%

%\RequirePackage{showframe}
% \if@showlayout
%         \RequirePackage{layout}
%         \AtEndDocument{%
%                 \clearpage
%                 \layout
%         }
% \fiifthenelse{\boolean{xetex} \or \boolean{luatex}}{}{%
% }



\PreventPackageFromLoading[%
\message{%
  ^^J^^JERROR: You tried to load a package that is not to be used with this class.^^J^^J}]{%
%
% Disallowed
%
% Layout
  geometry,a4,a4wide,%
% Fonts
  mathptmx,helvet,courier,newtxtext,newtxmath,fourier,%
%
  fancyhdr
%
% Not to use.
%
% Fonts / encoding
  ae,aecompl,zefonts,times,mathptm,pslatex,palatino,mathpple,%
  utopia,euler,%
  isolatin,umlaut,t1enc,%
% Graphics
  epsf,psfig,epsfig,subfig,subfigure%
% Outdated
  fancyheadings,scrpage,caption2,glossary,SIstyle,SIunits,%
% Typography
  doublespace%
}

\csgappto{bf}{\TR@fonterror}
\csgappto{it}{\TR@fonterror}
\csgappto{sc}{\TR@fonterror}
\csgappto{rm}{\TR@fonterror}
\csgappto{sc}{\TR@fonterror}
\csgappto{sf}{\TR@fonterror}
\csgappto{sl}{\TR@fonterror}
\csgappto{tt}{\TR@fonterror}

%% Warn about global use of sloppy
\let\TR@sloppy\sloppy
\patchcmd{\sloppypar}{\sloppy}{\TR@sloppy}{}{}
\patchcmd{\@arrayparboxrestore}{\sloppy}{\TR@sloppy}{}{}
\patchcmd{\thebibliography}{\sloppy}{\TR@sloppy}{}{}
\def\sloppy{\TR@warn{You should not use \string\sloppy. \MessageBreak
Instead, use a sloppypar, when necessary:\MessageBreak
\MessageBreak
\string\begin{sloppypar}\MessageBreak
\space\space...\MessageBreak
\string\end{sloppypar}}%
\TR@sloppy}



%%%%%%% Set counter to 5 (or v), respectively (See above)

\setcounter{page}{5}
\ifcsdef{frontmatter}{
  \apptocmd{\frontmatter}{%
    \setcounter{page}{5}%
  }{}{\TR@warn{Cannot patch \string\frontmatter}}
}{}


\RequirePackage{xspace}
\RequirePackage{csquotes}
\ifthenelse{\boolean{luatex}\OR\boolean{xetex}}{
  %% http://tex.stackexchange.com/a/16995
  \DeclareUTFcharacter[\UTFencname]{x201C}{\grqq}
  \DeclareUTFcharacter[\UTFencname]{x201E}{\glqq}
}{}
\PassOptionsToPackage{binary-units}{siunitx}
\AtEndPreamble{\RequirePackage{siunitx}}

%    \end{macrocode}
% \iffalse meta-comment
%</class>
% \fi^^A meta-comment
%
% \subsection{The Examples}
% \label{sec:The-Examples-and-the-Manual}
% \iffalse meta-comment
%<*example>
% \fi^^A meta-comment
%    \begin{macrocode}
\documentclass[%
  english,%
  todotools=true,%
%<proceedings>  trtype=proceedings%
%<single&article>  trtype=singlearticle%
%<single&!article>  trtype=singlereport,draft%
]{hpitr}
\usepackage{blindtext}
\usepackage{threeparttable}
\usepackage{listings}
%
% Pseudo-code example
%
\usepackage{algpseudocode}
\DeclareNewTOC[name=Algorithm,
  type=algorithm,
  atbegin=\KOMAoptions{captions=above},
  float]{alg}

\rowcolors*{2}{lightgray!25}{}

\begin{filecontents}{example.bib}
@article{953350,
  Address = {New York, NY, USA},
  Author = {Nassi, I. and Shneiderman, B.},
  Doi = {10.1145/953349.953350},
  Issn = {0362-1340},
  Journal = {SIGPLAN Not.},
  Number = {8},
  Pages = {12--26},
  Publisher = {ACM},
  Title = {Flowchart techniques for structured programming},
  Volume = {8},
  Year = {1973}}
\end{filecontents}

\begin{document}

% \authorrunning{Author}
% \titlerunning{The importance of work}
\title{The importance of why and how to do work}
\subtitle{An imaginary paper to showcase a document}
\author{Anna Author\and Bert Betatester}
\keywords{paper, showcase, lorem ipsum}
% \institute{%
%   Hasso-Plattner-Institut, Potsdam\\
%   \email{\{firstname.lastname\}@student.hpi.uni-potsdam.de}
% }
% \supervisors{%
%   Prof.\,Dr.\,William Withaname\and%
%  Dr.\,John Doe}
\maketitle

%<*!proceedings>
\begin{abstract}
  Hello, here is some text without a meaning. This text should show what a
  printed text will look like at this place. If you read this text, you will
  get no information. Really? Is there no information? Is there a difference
  between this text and some nonsense like “Huardest gefburn”? Kjift – not at
  all! A blind text like this gives you information about the selected font,
  how the letters are written and an impression of the look. This text should
  contain all letters of the alphabet and it should be written in of the
  original language. There is no need for special content, but the length of
  words should match the language.
\end{abstract}
%</!proceedings>
%<proceedings>\frontmatter
%<!article>\tableofcontents
%<!article>\lstlistoflistings
%<proceedings>\mainmatter

%<*!article>
\Blinddocument

\chapter{Other examples}
\label{cha:other}

Other sectioning examples, including listings and tables.

%</!article>
\section{Introduction}
\label{sec:introduction}

Hello, here is some text without a meaning. This text should show what a
printed text will look like at this place. If you read this text, you will get
no information. Really? Is there no information? Is there a difference between
this text and some nonsense like “Huardest gefburn”? Kjift – not at all! A
blind text like this gives you information about the selected font, how the
letters are written and an impression of the look. This text should contain all
letters of the alphabet and it should be written in of the original language.
There is no need for special content, but the length of words should match the
language.\todo{cite}

\subsection{Contributions}
\label{sec:contributions}


Hello, here is some text without a meaning. This text should show what a
printed text will look like at this place. If you read this text, you will get
no information. Really? Is there no information? Is there a difference between
this text and some nonsense like “Huardest gefburn”? Kjift – not at all! A
blind text like this gives you information about the selected font, how the
letters are written and an impression of the look. This text should contain all
letters of the alphabet and it should be written in of the original language.
There is no need for special content, but the length of words should match the
language.

\section{Context}
\label{sec:context}

Hello, here is some text without a meaning. This text should show what a
printed text will look like at this place. If you read this text, you will get
no information. Really? Is there no information? Is there a difference between
this text and some nonsense like “Huardest gefburn”? Kjift – not at all! A
blind text like this gives you information about the selected font, how the
letters are written and an impression of the look. This text should contain all
letters of the alphabet and it should be written in of the original language.
There is no need for special content, but the length of words should match the
language.

\todosec{Is that redundant?}

\subsection{Background}
\label{sec:background}

Hello, here is some text without a meaning. This text should show what a
printed text will look like at this place. If you read this text, you will get
no information. Really? Is there no information? Is there a difference between
this text and some nonsense like “Huardest gefburn”? Kjift – not at all! A
blind text like this gives you information about the selected font, how the
letters are written and an impression of the look. This text should contain all
letters of the alphabet and it should be written in of the original language.
There is no need for special content, but the length of words should match the
language.

\section{Problem}
\label{sec:problem}

\secmissing{The Main Problem}
Hello, here is some text without a meaning. This text should show what a
printed text will look like at this place. If you read this text, you will get
no information. Really? Is there no information? Is there a difference between
this text and some nonsense like “Huardest gefburn”? Kjift – not at all! A
blind text like this gives you information about the selected font, how the
letters are written and an impression of the look. This text should contain all
letters of the alphabet and it should be written in of the original language.
There is no need for special content, but the length of words should match the
language.~\cite{953350}



\subsection{Specific Problem}
\label{sec:specific-problem}

Hello, here is some text without a meaning. This text should show what a
printed text will look like at this place. If you read this text, you will get
no information. Really? Is there no information? Is there a difference between
this text and some nonsense like “Huardest gefburn”? Kjift – not at all! A
blind text like this gives you information about the selected font, how the
letters are written and an impression of the look. This text should contain all
letters of the alphabet and it should be written in of the original language.
There is no need for special content, but the length of words should match the
language.

\section{Solution}
\label{sec:solution}

\todolist{anton}{
\item Put A onto B
\item Put B into C
\item Pull D from C
}%
Hello, here is some text without a meaning. This text should show what a
printed text will look like at this place. If you read this text, you will get
no information. Really? Is there no information? Is there a difference between
this text and some nonsense like “Huardest gefburn”? Kjift – not at all! A
blind text like this gives you information about the selected font, how the
letters are written and an impression of the look. This text should contain all
letters of the alphabet and it should be written in of the original language.
There is no need for special content, but the length of words should match the
language.

\subsection{Specific Solution}
\label{sec:specific-solution}

Hello, here is some text without a meaning. This text should show what a
printed text will look like at this place. If you read this text, you will get
no information. Really? Is there no information? Is there a difference between
this text and some nonsense like “Huardest gefburn”? Kjift – not at all! A
blind text like this gives you information about the selected font, how the
letters are written and an impression of the look. This text should contain all
letters of the alphabet and it should be written in of the original language.
There is no need for special content, but the length of words should match the
language.


\begin{table}
  \centering
  \caption{Different Animals, different values}
  \label{tab:different}
  \begin{tabular}{@{}llSs@{}}
    \hiderowcolors
    \toprule
    \multicolumn{2}{c}{Item}\\ \cmidrule(r){1-2}
    Animal    & Description & {Value}   & Unit\\
    \midrule
    Gnat      & per gram    &    2.3456 & \dB                       \\
              & each        &    1.2e-3 & \metre\squared\per\second \\
    Gnu       & stuffed     &       e3  & \kilo\hertz               \\
    Emu       & stuffed     &   90.473  & \percent                  \\
    Armadillo & frozen      & 5642.5    & \mega\byte                \\
    \bottomrule
    \showrowcolors
  \end{tabular}
\end{table}

\section{Implementation}
\label{sec:implementation}

Hello, here is some text without a meaning. This text should show what a
printed text will look like at this place. If you read this text, you will get
no information. Really? Is there no information? Is there a difference between
this text and some nonsense like “Huardest gefburn”? Kjift – not at all! A
blind text like this gives you information about the selected font, how the
letters are written and an impression of the look. This text should contain all
letters of the alphabet and it should be written in of the original language.
There is no need for special content, but the length of words should match the
language.

\begin{figure}
  \centering
  \setlength{\unitlength}{.01in}%{.025in}
  \begin{picture}(200,75)
    \put(0,25){\vector(1,0){200}}
    \put(25,0){\vector(0,1){75}}
    \put(75,22){\line(0,1){6}}
    \put(125,22){\line(0,1){6}}
    \put(22,50){\line(1,0){6}}
    \thicklines
    \put(25,25){\line(1,0){50}}
    \put(75,50){\line(1,0){50}}
    \put(125,25){\line(1,0){72}}
    \put(17,50){\makebox(0,0){$1$}}
    \put(75,13){\makebox(0,0)[b]{$\pi$}}
    \put(125,13){\makebox(0,0)[b]{$2\pi$}}
    \put(195,13){\makebox(0,0)[b]{$t$}}
    \put(175,60){\makebox(0,0){$g(t)$}}
  \end{picture}
  \caption{A test figure}
  \label{fig:test}
\end{figure}

Hello, here is some text without a meaning. This text should show what a
printed text will look like at this place. If you read this text, you will get
no information. Really? Is there no information? Is there a difference between
this text and some nonsense like “Huardest gefburn”? Kjift – not at all! A
blind text like this gives you information about the selected font, how the
letters are written and an impression of the look. This text should contain all
letters of the alphabet and it should be written in of the original language.
There is no need for special content, but the length of words should match the
language.

\missingfigure{Make a overview sketch of the whole system}

\subsection{Minor Detail}
\label{sec:detail}

\begin{algorithm}
\caption{The Bellman-Kalaba algorithm}
\begin{algorithmic}[1]
\Procedure {BellmanKalaba}{$G$, $u$, $l$, $p$}
\ForAll {$v \in V(G)$}
\State $l(v) \leftarrow \infty$
\EndFor
\State $l(u) \leftarrow 0$
\Repeat
\For {$i \leftarrow 1, n$}
\State $min \leftarrow l(v_i)$
\For {$j \leftarrow 1, n$}
\If {$min > e(v_i, v_j) + l(v_j)$}
\State $min \leftarrow e(v_i, v_j)$\State $p(i) \leftarrow v_j$
\EndIf
$+l(v_j)$
\EndFor
\State $l’(i) \leftarrow min$
\EndFor
\State $changed \leftarrow l \not= l’$
\State $l \leftarrow l’$
\Until{$\neg changed$}
\EndProcedure
\Statex
\Procedure {FindPathBK}{$v$, $u$, $p$}
\If {$v = u$}
\State \textbf{Write} $v$
\Else
\State $w \leftarrow v$
\While {$w \not= u$}
\State \textbf{Write} $w$
\State $w \leftarrow p(w)$
\EndWhile
\EndIf
\EndProcedure
\end{algorithmic}
\end{algorithm}

Hello, here is some text without a meaning. This text should show what a
printed text will look like at this place. If you read this text, you will get
no information. Really? Is there no information? Is there a difference between
this text and some nonsense like “Huardest gefburn”? Kjift – not at all! A
blind text like this gives you information about the selected font, how the
letters are written and an impression of the look. This text should contain all
letters of the alphabet and it should be written in of the original language.
There is no need for special content, but the length of words should match the
language.

\section{Evaluation}
\label{sec:evaluation}

Hello, here is some text without a meaning. This text should show what a
printed text will look like at this place. If you read this text, you will get
no information. Really? Is there no information? Is there a difference between
this text and some nonsense like “Huardest gefburn”? Kjift – not at all! A
blind text like this gives you information about the selected font, how the
letters are written and an impression of the look. This text should contain all
letters of the alphabet and it should be written in of the original language.
There is no need for special content, but the length of words should match the
language.

\lstset{language=Java}
\begin{lstlisting}[label=lst:example,caption={An example code snippet},float,numbers=left]
class HelloWorldApp {
    public static void main(String[] args) {
        System.out.println("Hello World!"); // Display the string.
    }
}
\end{lstlisting}


We have nice things: some code in \autoref{lst:example} and some information
in \autoref{tbl:things}.

\[\nabla \cdot \mathbf{E} = \frac{\rho}{\varepsilon_0}\]
\[\nabla \cdot \mathbf{B} = 0\]
\[\nabla \times \mathbf{E} = -\frac {\partial \mathbf{B}}{\partial t}\]
\[\nabla \times \mathbf{B} = \mu_0 \mathbf{J} + \mu_0\varepsilon_0  \frac{\partial \mathbf{E}}{\partial t}\]


\subsection{Threats to Valitidy}
\label{sec:threats-valitidy}


Hello, here is some text without a meaning. This text should show what a
printed text will look like at this place. If you read this text, you will get
no information. Really? Is there no information? Is there a difference between
this text and some nonsense like “Huardest gefburn”? Kjift – not at all! A
blind text like this gives you information about the selected font, how the
letters are written and an impression of the look. This text should contain all
letters of the alphabet and it should be written in of the original language.
There is no need for special content, but the length of words should match the
language.

\section{Related Work}
\label{sec:related-work}

Hello, here is some text without a meaning. This text should show what a
printed text will look like at this place. If you read this text, you will get
no information. Really? Is there no information? Is there a difference between
this text and some nonsense like “Huardest gefburn”? Kjift – not at all! A
blind text like this gives you information about the selected font, how the
letters are written and an impression of the look. This text should contain all
letters of the alphabet and it should be written in of the original language.
There is no need for special content, but the length of words should match the
language.

Hello, here is some text without a meaning. This text should show what a
printed text will look like at this place. If you read this text, you will get
no information. Really? Is there no information? Is there a difference between
this text and some nonsense like “Huardest gefburn”? Kjift – not at all! A
blind text like this gives you information about the selected font, how the
letters are written and an impression of the look. This text should contain all
letters of the alphabet and it should be written in of the original language.
There is no need for special content, but the length of words should match the
language.


\section{Conclusion}
\label{sec:conclusion}

Hello, here is some text without a meaning. This text should show what a
printed text will look like at this place. If you read this text, you will get
no information. Really? Is there no information? Is there a difference between
this text and some nonsense like “Huardest gefburn”? Kjift – not at all! A
blind text like this gives you information about the selected font, how the
letters are written and an impression of the look. This text should contain all
letters of the alphabet and it should be written in of the original language.
There is no need for special content, but the length of words should match the
language.




\begin{table}
  \centering
  \begin{threeparttable}
    \caption{Differences between things projected and things achieved}
    \label{tbl:things}
    \begin{tabular}{>{}p{.4\linewidth}@{}c}
      \toprule
      Part           & done \\
      \midrule
      Title          & yes  \\
      Abstract       & no   \\
      Intro          & yes  \\
      \midrule
      \multicolumn{2}{c}{Rest is not entirely true} \\
      \midrule
      Context        & yes  \\
      Problem        & no\tnote{a} \\
      Solution       & yes  \\
      Implementation & yes  \\
      Evaluation     & no   \\
      Related Work   & no   \\
      Conclusion     & yes  \\
      \bottomrule
    \end{tabular}
    \begin{tablenotes}
    \item [a] Just a few things missing
    \end{tablenotes}
  \end{threeparttable}
\end{table}

We have nice things: some code in \autoref{lst:example} and some information
in \autoref{tbl:things}.

Hello, here is some text without a meaning. This text should show what a
printed text will look like at this place. If you read this text, you will get
no information. Really? Is there no information? Is there a difference between
this text and some nonsense like “Huardest gefburn”? Kjift – not at all! A
blind text like this gives you information about the selected font, how the
letters are written and an impression of the look. This text should contain all
letters of the alphabet and it should be written in of the original language.
There is no need for special content, but the length of words should match the
language.


\bibliography{example}

%%% Local Variables: 
%%% mode: latex
%%% End: 
\end{document}
%    \end{macrocode}
%
% \iffalse meta-comment
%</example>
% \fi^^A meta-comment
%
%
% \subsection{The Manual}
% \label{sec:manual}
%
% Load standard \LaTeX documentation class, passing all options to it
% but use |scrartcl| instead of |article|.
%    \begin{macrocode}
%<*doc|manual|README>
%<*!README>
\RequirePackage{scrlfile}
\ReplaceClass{article}{scrartcl}
%    \end{macrocode}
%
% Should we need \textsc{url}s, we want them to break nicely, also on hyphens.
% Since either \texttt{ltxdoc} or \texttt{hypdoc} loads the \texttt{url}
% package, we pass this option early.
%    \begin{macrocode}
\PassOptionsToPackage{hyphens}{url}
%    \end{macrocode}
% Load the \texttt{ltxdoc} class (patched to subsequently load a KOMAScript
% class). We set a calculating \texttt{DIV}\footnote{see KOMAScript guide} and
% the two-column option, since we eventually end up kind-of two-column anyways.
% However, we do this not by the normal means so we force |\onecolumn| once the
% document commences.
%    \begin{macrocode}
\documentclass{ltxdoc}
\KOMAoptions{%
  paper=a4,
  fontsize=9pt,
  pagesize,twocolumn=true,DIV=calc,mpinclude=true}
\AtBeginDocument{\onecolumn}
\usepackage{ifpdf,ifxetex,etoolbox,xspace}
\usepackage{color}
\usepackage[x11names,dvipsnames,svgnames]{xcolor}
%    \end{macrocode}
% Customize the display of our code:
% \begin{itemize}
% \item Use two columns at index
% \item Outdent a bit
% \item Make line numbers gray
% \item Print all code in monospaced font
% \end{itemize}
%    \begin{macrocode}
\setcounter{IndexColumns}{2}
\MacroTopsep=0pt
\MacrocodeTopsep=3pt
\setlength\MacroIndent{0pt}
\setlength{\columnsep}{18pt}
\providecommand*\theCodelineNo{%
  \normalfont\sffamily\color{black!50}\scriptsize%
  \arabic{CodelineNo}\ }
\let\dmsize\scriptsize
\def\dm#1{{\dmsize\textrm{\textit{\textcolor{DarkBlue}{#1}}}}\xspace}

\providecommand*\MacroFont{\ttfamily\small}
\providecommand*\AltMacroFont{\ttfamily\footnotesize}
\providecommand*\EnvName{environment}
\providecommand*\EnvsName{environments}
\providecommand*\OptionName{option}
\providecommand*\OptionsName{options}

 % tob be overridden by hypdoc
\renewcommand*\PrintMacroName[1]{\strut\MacroFont\string #1}
\renewcommand*\PrintDescribeMacro[1]{\strut \dm{c} \MacroFont #1\ }
\makeatletter
\renewcommand*\PrintEnvName[1]{%
  \HD@target%
  \strut \MacroFont #1\ }
\renewcommand*\PrintDescribeEnv[1]{%
  \HD@target%
  \strut \dm{e} \MacroFont #1\ }
\newcommand*\PrintDescribeOption[1]{%
  \HD@target%
  \strut \dm{o} \MacroFont #1\ }
\newcommand*\PrintOptionName[1]{%
  \HD@target%
  \strut \MacroFont #1\ }
\makeatother
\newcommand*\PrintDescribeValue[2]{%
  \strut \dm{v} \MacroFont{\scriptsize #1=}#2\ }


\usepackage{hypdoc}

\makeatletter
\def\macro{\begingroup
  \catcode`\\12
  \MakePrivateLetters \m@cro@{0}}
\def\environment{\begingroup
  \catcode`\\12
  \MakePrivateLetters \m@cro@{1}}
\def\option{\begingroup
  \catcode`\\12
  \MakePrivateLetters \m@cro@{2}}

%^^A \def\endmacro{}
\long\def\m@cro@#1#2{\endgroup \topsep\MacroTopsep \trivlist
  \edef\saved@macroname{\string#2}%
  \def\makelabel##1{\llap{##1}}%
  \if@inlabel
    \let\@tempa\@empty \count@\macro@cnt
    \loop \ifnum\count@>\z@
      \edef\@tempa{\@tempa\hbox{\strut}}\advance\count@\m@ne 
    \repeat
    \edef\makelabel##1{\llap{\vtop to\baselineskip
        {\@tempa\hbox{##1}\vss}}}%
    % \def\makelabel##1{\\##1}
    \advance \macro@cnt \@ne
  \else \macro@cnt\@ne \fi
  \edef\@tempa{\noexpand\item[%
    \ifcase #1
      \noexpand\PrintMacroName
    \or
      \noexpand\PrintEnvName
    \or
      \noexpand\PrintOptionName
    \fi
    {\string#2}]%
  }%
  \@tempa
  \global\advance\c@CodelineNo\@ne
  \ifcase #1
    \SpecialMainIndex{#2}\nobreak
    \DoNotIndex{#2}%
  \or
    \SpecialMainEnvIndex{#2}\nobreak
  \or
    \SpecialMainOptionIndex{#2}\nobreak
  \fi
  \global\advance\c@CodelineNo\m@ne
  \ignorespaces}
\let\endoption\endmacro

% Defining new main index commands
\newcommand*{\SpecialMainIndex@Type}[3]{%
  \@bsphack\special@index{%
    #1\actualchar
    {\string\ttfamily\space#1}
    (\string #2)%
    \encapchar main}%
  \special@index{%
    #3:\levelchar{%
      \string\ttfamily\space#1}\encapchar
    main}\@esphack}
\renewcommand*{\SpecialMainEnvIndex}[1]{%
  \SpecialMainIndex@Type{#1}{\EnvName}{\EnvsName}}
\newcommand*{\SpecialMainOptionIndex}[1]{%
  \SpecialMainIndex@Type{#1}{\OptionName}{\OptionsName}}



% Defining new usage index commands
\newcommand*{\SpecialIndex@Type}[3]{%
  \@bsphack
  \begingroup
    \HD@target
    \let\HDorg@encapchar\encapchar
    \edef\encapchar usage{%
      \HDorg@encapchar hdclindex{\the\c@HD@hypercount}{usage}%
    }%
  \index{#1\actualchar{\protect\ttfamily#1}
    (#2)\encapchar usage}%
  \index{#3:\levelchar{\protect\ttfamily#1}\encapchar
    usage}\endgroup\@esphack}
\renewcommand{\SpecialEnvIndex}[1]{%
  \SpecialIndex@Type{#1}{\EnvName}{\EnvsName}}
\newcommand*{\SpecialOptionIndex}[1]{%
  \SpecialMainIndex@Type{#1}{\OptionName}{\OptionsName}}

\newbool{TR@indescribe}
\def\TR@describe@table@begin{%
  \ifvmode\else\par\fi\small\addvspace{2\baselineskip}%
  \vspace*{-\baselineskip}%
  \vspace{\z@ plus \baselineskip}%
  \noindent%
  \hspace{-1em}
  \tabular{|l|}\hline\ignorespaces}
\def\TR@describe@table@end{%
  \\%
  \hline\endtabular\nobreak\par\nobreak
  \vspace{1.5\baselineskip}\nobreak\vspace{-\baselineskip}\nobreak%
  \vspace{0pt minus .5\baselineskip}\nobreak
}

% Define new describe commands
\newcommand*{\newDescribe}[1]{%
  \csgdef{Describe#1}{%
    \leavevmode\@bsphack\begingroup\MakePrivateLetters%
    \csuse{Describe@#1}}%
  \csgdef{Describe@#1}##1{%
    \endgroup%
    \ifthenelse{\boolean{TR@indescribe}}{}{\TR@describe@table@begin}%
    \noindent\mbox{\csuse{PrintDescribe#1}{##1}}%
    \ifthenelse{\boolean{TR@indescribe}}{}{\TR@describe@table@end}%
    \csuse{Special#1Index}{##1}\@esphack\ignorespaces}%
}
\newDescribe{Env}
\newDescribe{Option}
\def\Describe@Macro#1{\endgroup%
  \ifthenelse{\boolean{TR@indescribe}}{}{\TR@describe@table@begin}%
  \noindent\mbox{\PrintDescribeMacro{#1}}%
  \ifthenelse{\boolean{TR@indescribe}}{}{\TR@describe@table@end}%
  \SpecialUsageIndex{#1}\@esphack\ignorespaces}

\def\DescribeValue{\leavevmode\@bsphack\begingroup%
  \MakePrivateLetters\Describe@Value}
\def\Describe@Value#1#2{%
    \endgroup%
    \ifthenelse{\boolean{TR@indescribe}}{}{\TR@describe@table@begin}%
    \noindent\mbox{\PrintDescribeValue{#1}{#2}}%
    \ifthenelse{\boolean{TR@indescribe}}{}{\TR@describe@table@end}%
    \@esphack\ignorespaces}%

\newenvironment{Describe}{%
  \ifthenelse{\boolean{TR@indescribe}}{\error{Do not nest Describe}}{}%
  \booltrue{TR@indescribe}%
  \TR@describe@table@begin%
}{%
  \TR@describe@table@end%
  \boolfalse{TR@indescribe}%
  \nobreak%
  \ignorespaces%
}
\makeatother


%    \end{macrocode}
% Packages we want for hyper-reference support, a tight table of contents,
% fonts (EB~Garamond and the Adobe Source family), and finally a package for
% parallel typesetting, among others.
%    \begin{macrocode}
\usepackage[tocflat,toctextentriesindented]{tocstyle}%
\usetocstyle{nopagecolumn}
\usepackage[toc]{multitoc}
\usepackage{fontspec}
\usepackage[scale=MatchLowercase]{ebgaramond}
\usepackage[scale=MatchLowercase,semibold,osf]{sourcesanspro}
\usepackage[scale=MatchLowercase,semibold]{sourcecodepro}
\RequirePackage{hyphsubst}
\HyphSubstLet{ngerman}{ngerman-x-latest}
\usepackage[main=english,ngerman]{babel}
\babeltags{german=ngerman,english=english}
\usepackage{graphicx}
\usepackage{booktabs}
\usepackage{ragged2e}
\usepackage{array}
\usepackage[binary-units]{siunitx}
\DeclareSIUnit[number-unit-product = {}] \pt{pt}
\usepackage{listings}
%^^A\definecolor{Blue1}{rgb}{0,0,1}
\lstdefinelanguage[hpitr]{TeX}[AlLaTeX]{TeX}{%
  deletetexcs={title,author},
  morekeywords={abstract},
  moretexcs={chapter},
  moretexcs=[2]{title,author,subtitle,keywords,maketitle},
  moretexcs=[3]{addbibresource,printbibliography},
}
\lstset{%
  basicstyle={\color{darkgray}\small\ttfamily},%
  language={[hpitr]TeX},%
  columns=fullflexible,%
  stringstyle=\color{RosyBrown},%
  texcsstyle=*{\color{Purple}\mdseries},%
  texcsstyle=*[2]{\color{Blue1}},%
  texcsstyle=*[3]{\color{ForestGreen}},%
  % identifierstyle=\color{Blue1},%
  commentstyle={\color{FireBrick}},%
  escapechar=`,%
}
\usepackage{threeparttable}
\usepackage{paracol}
\marginparthreshold{0}
%    \end{macrocode}
% We predominantly want to use the set serif font for all sectioning. We also
% want a good text area.
%    \begin{macrocode}
\setkomafont{subtitle}{\usekomafont{title}\LARGE\em}
\addtokomafont{disposition}{\rmfamily}

\providecommand*\generalshape{\upshape}
\setkomafont{title}{\rmfamily\mdseries}
\addtokomafont{subject}{\mdseries}
\setkomafont{disposition}{\rmfamily\mdseries\generalshape}
\addtokomafont{footnote}{\generalshape}
\addtokomafont{paragraph}{\em}
\addtokomafont{descriptionlabel}{\mdseries}
\addtokomafont{captionlabel}{\mdseries}
\renewcommand*\labelitemii{\normalfont\textendash}


\renewenvironment{abstract}%
{\list{}{\listparindent 1em%
    \advance\leftmargin 4em\relax
    \itemindent    \listparindent
    \rightmargin   \leftmargin
    \parsep        0pt plus 1pt\relax
  }%
  \item\relax}
{\endlist}

\setfootnoterule{4cc}
\hypersetup{%
    colorlinks=true,   %% true for printout
    linkcolor=MidnightBlue,%
    citecolor=MidnightBlue,%
    urlcolor=MidnightBlue,%
}
\raggedbottom
\recalctypearea
%    \end{macrocode}
% For debugging only
%    \begin{macrocode}
% \usepackage{showframe}
% \errorcontextlines=999
%    \end{macrocode}
% A semantic macro for files and paths. (\emph{Note:} cannot be a |\let|, or
% else |hypdoc| is upset.)
%    \begin{macrocode}
\newcommand*{\File}{}\def\File{\texttt}

\AtBeginDocument{\addcontentsonly{toc}{0}}
\footnotelayout{m}
\newenvironment{eng}[1][]%
  {\begin{leftcolumn*}[#1]\begin{english}%
        \let\generalshape\upshape%
      }%
  {\end{english}\end{leftcolumn*}}
\newenvironment*{ger}%
{\begin{rightcolumn}
 \begin{german}%
   \let\generalshape\em%
   \let\origtheHsection\theHsection%
   \def\theHsection{ignore.\origtheHsection}%
   \em}%
{\end{german}%
 \end{rightcolumn}}

\renewcommand*\listoftocname{%
  \textenglish{Contents} / \emph{\textgerman{Inhalt}}}
\let\contentsname\mycontentsname

\newcommand\remark[1]{%
  \unskip\par\vspace{\baselineskip}
  \noindent\framebox[\linewidth]{%
    \parbox{\linewidth}{\centering
    #1%
  }}%
  \vspace{\baselineskip}\par%
}
\usepackage{enumitem}
\setlist{noitemsep,topsep=.5\baselineskip,leftmargin=0pt}
\setdescription{noitemsep,font=\normalfont\scshape,leftmargin=\parindent}
\DeclareRobustCommand\cf[1]{%
cf.\,{#1}
}
\GlossaryPrologue{%
  \section*{\textenglish{Change History} / \emph{\textgerman{Änderungsverlauf}} }}
\IndexPrologue{
\begin{paracol}{2}[\section*{Index}%
  \markboth{Index}{Index}]
\begin{eng}
  Numbers written in italic refer to the page where the corresponding entry is
  described; numbers underlined refer to the code line of the definition;
  numbers in roman refer to the code lines where the entry is used.
\end{eng}\begin{ger}
  Kursive Zahlen bezeichnen die Seite, auf welcher ein Eintrag beschrieben ist;
  unterstrichene Zahlen bezeichnen die Quelltextzeile seine Definition;
  nicht-ausgezeichnete Zahlen beziehen sich auf Quelltextzeilen, wo der Eintrag
  benutzt wird.
\end{ger}
\end{paracol}
}


%<!manual>\CodelineIndex
\RecordChanges
\EnableCrossrefs
%<manual>\OnlyDescription
\GetFileInfo{\jobname.ltx}
\begin{document}
\title{The Class \File{hpitr} for Writing Technical Reports at the
  Hasso~Plattner~Institute\footnote{This is file version \fileversion{} of
    file \File{\filename}.}}
\subtitle{Die Klasse \File{hpitr} zur Erstellung technischer Berichte am
  Hasso-Plattner-Institut}
\author{Tobias Pape\footnote{Tobias Pape
    \textless\protect\url{tobias.pape@hpi.de}\textgreater}}
\date{\fileversion{} \filedate}
\maketitle
\begin{abstract}
%</!README>
%<*!manual>

%<!README>\begin{paracol}{2}[]\begin{eng}
%<!README>This consititutes the \LaTeX\ class to creat technical reports at the 
%<README>This consititutes the LaTeX class to creat technical reports at the 
Plattner Institute, Potsdam in conjunction with the Universitätsverlag
Potsdam. To maintain a unified appearance, this class provides macrotypographic
(like paper size and general layout) and microtypographic (like fonts and their
adjustment) settings.
%<!README>\end{eng}

%<README> ---

%<!README>\begin{ger}
%<!README>Diese \LaTeX-Klasse dient zur Erstellung von technischen Berichten
%<README>Diese LaTeX-Klasse dient zur Erstellung von technischen Berichten
am Hasso-Plattner-Institut, Potsdam, in Zusammenarbeit mit dem
Universitätsverlag Potsdam. Zur Erhaltung eines einheitlichen
Erscheinungsbildes bietet diese Klasses Einstellungen zu Makrotypographie (wie
Papiergröße, allgemeines Layout) und auch zur Detailtypographie (wie
Schriftwahl).
%<!README>\end{ger}\end{paracol}


%</!manual>
% \end{macrocode}
%
% The installation instructions that end up in the README are from the
% \texttt{titlepage.dtx} again.
% \begin{macrocode}
%<*README>

-------------------------------------------------------------------------------
Note: To generate all files, you should simply call

      luatex -shell-escape hpitr.dtx

If you are using MiKTeX you have to use

     luatex --enable-write18 hpitr.dtx

to fully enable shell escapes which are also known as \write18 feature.

All ltx files should be installed together with all pdf files at documentation
folder. The sty file together with all def files should be installed at latex
package folder.

But maybe your distributor already distributes a ready for installation
package, so you do not need to create files and copy them yourself.
%</README>
%<*manual>
This manual gives a concise usage guide for the \texttt{hpitr} class, used to
prepare technical reports at the Hasso Plattner Institute, Potsdam
in conjunction with the Universitätsverlag Potsdam. We describe newly
introduced commands and environments and document deviations from what may seem
typical, like non-standard paper size.
%</manual>
%<*!README>
\end{abstract}
%<*manual>
%^^A ==========================================================================
%^^A ==========================================================================
%^^A ==========================================================================
\tableofcontents
\iffalse
TODO:

1. Benutze, wenn möglich, LuaLaTeX oder XeLaTeX (lieber lualatex). Die Uni möchte gerne 
  Unicode-Schriften und das ist mit reinem pdfLaTeX nicht zu machen. Außerdem können
  beide nativ UTF-8:
2. Bitte nutze ausschließlich utf-8-codierte .tex dateien (nicht latin1 oder ansi-new)

- microtype, graphicx, amsmath, amssymb, url, booktabs, color, fixltx2e und hyperref
 Werden von hpitr schon geladen, man kann sie also aufführen, muss aber nicht.
- times, subfig
 Sind veraltet und times ist außerdem mit hpitr nicht zulässig
- ctable, wrapfix, algorithmic, algorithm2e
 Werden gar nicht verwendet, also habe ich sie entfernt sodass sie nicht stören.
- WICHTIG: inputenc
 Habe ich entfernt, da ich deinen TR mit lualatex gebaut habe. 
 Außerdem ist mit hpitr nur utf-8 zulässig:
   - Wenn man mit pdflatex baut, ruft hpitr automatisch \cs\usepackage[utf8]{inputenc} auf.

 Nebenbei: 
	%\cs\usepackage[latin1]{inputenc} %Linux
	\cs\usepackage[ansinew]{inputenc} %Windows
	%\cs\usepackage[applemac]{inputenc} %Mac
 Sollte in deinen Dokumenten nicht auftauchen.
 Denn zumeist stimmt dass nicht mehr. Auf den meisten Linux'en ist
 nicht mehr latin-1, sondern utf-8 der standard. Bei windows kommt es aufs
 land an. Auf dem mach wird seit 14 jahren kein 'applemac' mehr verwendet.
 Auch hier wäre utf-8 besser.
 :)


Die Änderungen and labelitemi... verändern das Geseamtaussehen und
sind mit hpitr nicht zulässig.
%%%%%

title/author
Hier ist der Titel von oben. hpitr trägt diese werte automatisch in das pdf ein (und kann auch mit \\ im titel umgehen)



%%%%%
For conversions from llncs:

Du benutzt die Defintion-umgebung. Die gibts in lncs, aber nicht
sonst. Ich habe sie dir mit amsthm nachgebaut.

\fi

\begin{paracol}{2}
\begin{eng}
\section{Introduction}
\label{sec:intro}
All technical reports the Hasso-Plattner-Institut für Sofwaresystemtechnik (\textsc{hpi})
issues should have a uniform appearance. The \LaTeX{} class \File{hpitr} exists
for this very purpose. Its aim is to ease the work with and on technical
reports of the \textsc{hpi} and ensure their uniformity.

This manual illustrates the means this \LaTeX{} class provides and explains the
necessary constraints.
\end{eng}
\begin{ger}
\section{Einführung}
Alle technischen Berichte, die das Hasso"-Plattner"-Institut für
Sofwaresystemtechnik (\textsc{hpi}) herausgibt sollen ein einheitliches
Erscheinungsbild haben. Zu diesem Zweck existiert diese \LaTeX"=Klasse names
\File{hpitr}. Sie soll die Arbeit an und mit technischen Berichten des
\textsc{hpi} vereinfachen und die Einheitlichkeit sicherstellen.

Im diesem Handbuch werden die Möglichkeiten dieser \LaTeX"=Klasse erläutert
und die notwendigen Einschränkungen erklärt.
\end{ger}
\end{paracol}

\begin{paracol}{2}
\begin{eng}
\subsection{A note on \LaTeX{} elements}
\end{eng}
\begin{ger}
\subsection{Eine Anmerkung zu \LaTeX"=Befehlen} 
\end{ger}
\end{paracol}
\begin{Describe}
  \DescribeMacro{command}\\
  \DescribeEnv{environment}\\
  \DescribeOption{option}\\
  \DescribeValue{option}{value}
\end{Describe}
\begingroup \let\dmsize\normalsize
\begin{paracol}{2}
\begin{eng}
To easily find documentation to specific \LaTeX{} elements this class
provides, that location is highlighted by a box like above. Every \LaTeX{}
element is prefixed by its type: command sequences \dm{c}, environments
\dm{e}, class options \dm{o}, and values to class option keys \dm{v}. This
should help navigating in the  manual.
\end{eng}
\begin{ger}
  Zur schnellen Hilfe zu bestimmen \LaTeX"=Befehlen, die von dieser Klasse
  angeboten werden, wird die zugehörige Erklärung befindet, mit einem Kasten
  wie oben zusehen markiert. Die einzelnen Befehlsartenhaben dabei eigene
  Präfixe: Befehle~\dm{o}, Umgebungen~\dm{e}, \mbox{Klassenoptionen~\dm{o}}
  und Werte für Klassenoptionsschlüssel~\dm{v}. Damit soll das Zurechtfinden
  im Handbuch erleichtert werden.
\end{ger}
\end{paracol}
\endgroup

\begin{paracol}{2}
\begin{eng}
\subsection{Manual Structure}
A \hyperref[sec:example]{short example report} follows this section. How to
compile a \File{hpitr} report is explained in \hyperref[sec:running]{the
section after that}. After
that, the different types of reports, necessary information about the report,
general typesetting requirements, and specific microtypography are
considered; followed by notes on the bibliographic style.
\end{eng}
\begin{ger}
\subsection{Aufbau}
Es folgt \hyperref[sec:example]{ein kurzer Beispielbericht}. Wie die Berichte
erzeugt werden wird im \hyperref[sec:running]{darrauffolgenden Abschnitt}
erklärt. Danach wird auf Berichtarten, notwendige Berichtinformationen,
generelle Satzeigenschaften, und detailtypographische Besonderheiten
eingegangen. Dann folgen Anmerkungen zum Zitierstil.
\end{ger}
\end{paracol}

\begin{paracol}{2}
\begin{eng}
\section{An Example}
\label{sec:example}
As jump-start, a very minimal but complete \File{hpitr} technical report follows:
\begin{lstlisting}
\documentclass[english,trtype=singlereport]{hpitr}
% Use Biblatex
\usepackage[backend=biber]{biblatex}
\addbibresource{references.bib}

\begin{document}
\title{The importance of why and how to do work}
\subtitle{An imaginary paper}
\author{Anna Author\and Bert Betatester}
\keywords{paper, showcase, lorem ipsum}
\maketitle

\begin{abstract}
  `This “paper”…`
\end{abstract}
\chapter{Introduction}
In computer science~\cite{myref}...
\chapter{...}
...
\printbibliography
\end{document}
\end{lstlisting}
This example sets a \File{hpitr} report that
 \begin{itemize}
 \item is an original report (\lstinline|trtype=singlereport|),
 \item uses Biblatex,
 \item and gives all meta-information for title, author, and keywords.
\end{itemize}
Note that by intention, the typical selection of input encoding
(\lstinline|\usepackage[...]{inputenc}|) is not present in this example. All
\File{hpitr} reports are to be written using \textsc{utf-8} encoding. The class
file takes care to select the right input encoding.

\remark{All files of a \File{hpitr} report must be \textsc{utf-8}
encoded.}

\end{eng}
\begin{ger}
\section{Ein Beispiel}
Als Starthilfe folg ein minimaler, doch vollständiger Technischer Bericht mit
\File{hpitr}:
\begin{lstlisting}
\documentclass[german,trtype=singlereport]{hpitr}
% Biblatex benutzen
\usepackage[backend=biber]{biblatex}
\addbibresource{references.bib}

\begin{document}
\title{Von der Wichtigkeit der Arbeit und sie zu tun}
\subtitle{Eine imaginäre Abhandlung}
\author{Anna Author\and Bert Betatester}
\keywords{Arbeit, Beispiel, lorem ipsum}
\maketitle

\begin{abstract}
  `Diese „Arbeit“...`
\end{abstract}
\chapter{Einführung}
Die Informatik hat~\cite{myref}...
\chapter{...}
...
\printbibliography
\end{document}
\end{lstlisting}
Dieses Beispiel erzeugt einen \File{hpitr}"=Bericht, der
\begin{itemize}
\item ein eigentlicher bericht ist (\lstinline|trtype=singlereport|),
\item Biblatex benutzt
\item und alle Meta"=Informationen für Title, Autoren und Schlagwörter
  bereitstellt.
\end{itemize}
Zu beachten ist, dass die eigentlich typische Auswahl der Eingabecodiereung
(\lstinline|\usepackage[...]{inputenc}|) im Beispiel nicht vorhanden ist. Dies
geschieht absichtilich, das alle Dateien eines \File{hpitr}"=Berichts in
\textsc{utf-8}"=Codierung geschrieben werden sollen. Die Klasse sorgt dann für
die korrerte Handhabung.

\remark{Alle Dateien eines \File{hpitr}"=Berichts müssen
  \textsc{utf-8}"=codiert sein.}
\end{ger}
\end{paracol}
%^^A ----------------------------------------
\begin{paracol}{2}
\begin{eng}
\section{Running \LaTeX}
\label{sec:running}
\File{hpitr} reports are handed to the Univerlag publisher as \textsc{pdf}
files. 
\end{eng}
\begin{ger}
\section{\LaTeX"=Läufe}
test
\end{ger}
\end{paracol}
%^^A----------------------------------------
\begin{paracol}{2}
\begin{eng}
\section{Types of Reports}
\label{sec:types}
\end{eng}
\begin{ger}
\section{Berichtarten}
\end{ger}
\end{paracol}
\DescribeOption{trtype=\meta{type}}
\begin{paracol}{2}
\begin{eng}
This class support the types of technical reports that are common at the HPI.
In essence, there are three variants.
\end{eng}
\begin{ger}
Die Klasse unterstütz die am HPI üblichen Arten von technischen Berichten.
Grob sind das drei Varianten.
\end{ger}
\end{paracol}


\begin{Describe}
  \DescribeValue{trtype}{singlearticle}\\
  \DescribeValue{trtype}{article}
\end{Describe}
\begin{paracol}{2}
\begin{eng}
\paragraph{Article-like reports}
\label{sec:art-like}
Reports that are basically similar to conference or journal articles or
extend such articles have an \emph{article-like} character. Their top-level
sectioning element is the \emph{sections}.
\end{eng}
\begin{ger}
\paragraph{Artikelartige Berichte}
Berichte, die im Kern ähnlich zu Tagungs-, Konferenz- oder
Zeitschriftenartikeln sind, oder erweiterte Varianten davon sind, haben
\emph{artikelartigen} Charakter. Ihre höchste Gliederungsebene sind
\emph{Abschnitte}.
\end{ger}
\end{paracol}


\begin{Describe}
\DescribeValue{trtype}{singlereport}\\
\DescribeValue{trtype}{report}
\end{Describe}
\begin{paracol}{2}
\begin{eng}
\paragraph{Original reports}
\label{sec:original-rprt}
Reports that are like monographs (such as theses) or inherently independent
reports are report in the strict sense.\footnote{Not implying that the other
types are inadequate as technical reports, to the contrary!} Their top-level
sectioning element is the \emph{chapter}.
\end{eng}
\begin{ger}
\paragraph{Eigentliche Berichte}
Berichte, die ähnlich zu Monographien wie Abschlussarbeiten sind oder
grundsätzlich eigenständige Berichte sind im eigentlichen Sinne
\emph{berichtartig}.\footnote{Das bedeutet nicht, dass die anderen Arten sich
nicht als technische Berichte eignen.} Ihre höchste Gliederungsebene sind
\emph{Kapitel}.
\end{ger}
\end{paracol}


\begin{Describe}
  \DescribeValue{trtype}{collection}\\
  \DescribeValue{trtype}{proceedings}
\end{Describe}
\begin{paracol}{2}
\begin{eng}
\paragraph{Collection-like reports}
\label{sec:coll-like}
Reports comprising several individual contributions are
\emph{collection-like}. This may include conference proceedings, seminar
compilations, or special issues, among others. Their top-level sectioning
element is the \emph{chapter}. Some commands to
manage such collections are provided for convenience.

Note that proceeding use a different formatting for the individual
contributions. They start with their own title, author information, and
abstract, while in normal collections, this information is absent. In normal
collections, all sections are numbered by chapter.
\end{eng}
\begin{ger}
\paragraph{Sammlungsartige Berichte}
Berichte, die sich aus mehreren Einzelbeiträgen zusammensetzen sind
\emph{sammlungsartig}. Dies kann unter anderem Tagungsbände,
Seminarzusammenfassungen oder Sonderbände beinhalten. Ihre höchste
Gliederungsebene sind \emph{Kapitel}. Es gibt einige Befehle die die
Verwaltung von diesen Sammlungen vereinfachen sollen.

Tagungsbände zeichen sich durch eine besondere Formatierung der
Einzelbeiträge aus; jeder Beitrag beginnt im Gegensatz zu normalen Sammlungen
mit eigenem Titel, Autorennennung und Zusammenfassung. In normalen Sammlungen
ordnet sich die Abschnitsnummerierung der Kapitelnummerierung unter.%
\end{ger}
\end{paracol}

\begin{paracol}{2}
\begin{eng}
\section{Report Information and Metadata}
\label{sec:metadata}
\begin{itemize}
\item Title, name, etc. pp
\item NO DOC TITLE
\item Special considerations for collection work.
\item --> authorship of individual contributions
\item Examples
\item LANGUAGE (+ oxford comma)
\end{itemize}

\end{eng}
\begin{ger}
\section{Informationen zum Bericht und Metadaten}
\end{ger}
\end{paracol}
%^^A------------------------
\begin{paracol}{2}
\begin{eng}
\section{General Layout}
\label{sec:general}
The technical reports of the \textsc{hpi} are published both in printed form
and online. To retain a uniform appearance across those media \emph{and}
across all supported report types, all report have to adhere to the same
layout.

\subsection{Page and type area}
\begin{itemize}
\item All text is set in a single column.
\item The paper size is \SI{210}{\mm}\(\times\)\SI{297}{\mm}
  (\textsc{iso}\,216 (\textsc{din}) A4).
\item There should be typically not more than 80 characters per line.
\item The margins fulfill the following equations: {\let\V\mathit
  \[ \V{top} < \V{left} \leq \V{right} < \V{bottom} \text{ and } \V{top} : \V{bottom} = 1 : 2 \]
  }
\end{itemize}
To ensure this layout, the class chooses (approximately) the side margins as
\SI{31.5}{\mm}, the top margin \SI{29.5}{\mm} and the bottom margin as
\SI{54}{\mm}. Note that the top margin is slightly larger than half the
bottom margin to optically respect the quite filled page header. The actual
computation of the type area is described in TODO:
REF.\footnote{\lstset{basicstyle=\scriptsize\ttfamily}The relevant options to
\File{typearea} are \lstinline|paper=a4,DIV=calc,twoside=semi| and
\lstinline|footinclude=false,headinclue=true,fontsize=12pt|}.

Page numbers are set centered on the page foot.

\remark{The page dimensions, margins, and type area must not be changed.}

\end{eng}
\begin{ger}
\section{Allgemeine Satzeigenschaften}
Technische Berichte des \textsc{hpi} werden sowohl gedruckt als auch
elektronisch verlegt. Um ein einheitliches Erscheinungsbild über alle Medien
und Berichtarten hinweg zu gewährleisten, müssen alle Berichte den gleichen
Satzeigenschaften genügen.

subsection{Seite und Satzspiegel}
\begin{itemize}
\item Der Satz erfolgt ausschließlich einspaltig.
\item Die Papiergröße ist \SI{210}{\mm}\(\times\)\SI{297}{\mm}
(\textsc{iso}\,216 (\textsc{din}) A4).
\item Die Zeilenlänge sollte 80 Zeichen nicht überschreiten.
\item Die Stege (Ränder) sollten folgenden Gleichungen genügen:{\let\V\mathit
  \[ \V{oben} < \V{links} \leq \V{rechts} < \V{unten} \text{ und } \V{oben} : \V{unten} = 1 : 2 \]
  }
\end{itemize}
Um diesen Satzspiegel zu gewährleisten, wählt die Klasse (ungefähr) die
Seitenstege mit \SI{31.5}{\mm}, den Kopfsteg mit \SI{29.5}{\mm} und den
Fußsteg mit \SI{54}{\mm}. Das genaue \(1:2\)"=Verhältnis zwischen Kopf"~ und
Fußsteg wird absichtlich leicht verletzt, um der recht vollen Kopfzeile
optisch Rechnung zu tragen. Die eigentliche Satzspiegelkonstruktion ist in
TODO:REF beschrieben.\footnote{\lstset{basicstyle=\scriptsize\ttfamily} Die
relevanten Werte für \File{typearea} sind
\lstinline|paper=a4,DIV=calc,twoside=semi| und
\lstinline|footinclude=false,headinclue=true,fontsize=12pt|}

Die Paginierung (Seitenzahlen) erfolgt zentriert im Seitenfuß.

\remark{Seitengröße, Stege (Ränder) und Satzspiegel dürfen nicht verändert
werden.}

\end{ger}
\end{paracol}
\begin{paracol}{2}
\begin{eng}
\subsection{Titles, Document Parts, and Page Numbering}
\label{sec:title}
\begin{itemize}
\item There is no title
\item Starts at 5 (or v with frontmatter stuff)
\end{itemize}
\end{eng}
\begin{ger}
\subsection{Titelei, Dokumentteile und Seitennummerierung}
\end{ger}
\end{paracol}
\begin{paracol}{2}
\begin{eng}
\subsection{Floats}
\label{sec:floats}
\end{eng}
\begin{ger}
\subsection{Fließumgebungen}
\end{ger}
\end{paracol}

\begin{paracol}{2}
\begin{eng}
\section{Microtypography}
\label{sec:micro}
\begin{itemize}
\item Font, font size, line spread,
\item avoid bold, use italics and small caps.
\item Abbreviations (ie ...)
\item 
\end{itemize}
\end{eng}
\begin{ger}
\section{Detailtypographie}
test
\end{ger}
\end{paracol}

\begin{paracol}{2}
\begin{eng}
\section{Bibliographic Style}
\label{sec:biblio}
For citations and bibliographies, a \emph{numeric} citation style is used.
The numbers derive from the \emph{alphabetic} sorting of the list of
bibliographic entries at the end of the report (or the end of the individual
contributions in collection works, if applicable). Given names are
abbreviated.

\begingroup\normalfont
\vspace*{\baselineskip}
\parbox[t]{1.5em}{[1]}\parbox[t]{\dimexpr\linewidth - 2em}{%
I.~Nassi and B.~Shneiderman.
\newblock Flowchart techniques for structured programming.
\newblock {\em SIGPLAN Not.}, 8(8):12--26, 1973.}
\vspace*{\baselineskip}
\endgroup

This corresponds to the \emph{plain} \BibTeX style, or the
\emph{numeric-comp} style with abbreviations for Biblatex. Either way, the
citation style is preselected. If in doubt, the Biblatex variant should be
used.

\remark{The citation and bibliographic style is not to
be changed.}

While \BibTeX should work fine and is supported, the use of Biblatex,
especially with its \emph{biber} backend is encouraged to support proper use of
Unicode characters and fonts. %^^A TODO: ref biblatex
Collection works \emph{must} use Biblatex. For a list of kinds of works to cite
as supported by \BibTeX and Biblatex refer to \autoref{tbl:bibtytpes}. Kinds of
work not listed should use the \emph{misc} type.


\end{eng}
\begin{ger}
\section{Zitier"~ und Literaturstil}
In Zitaten und Literaturlisten wird ein einfacher, \emph{numerischer} Stil
genutzt. Die Nummerierung folgt der \emph{alphabetischen} Sortierung der
Literatureinträge in der Literaturliste am Ende des Berichts (oder am Ende
der Einzelbeiträge in sammlungsartigen Berichten, sofern angebracht).
Vornamen werden abgekürzt.

\begingroup\normalfont
\vspace*{\baselineskip}
\parbox[t]{1.5em}{[1]}\parbox[t]{\dimexpr\linewidth - 2em}{%
I.~Nassi and B.~Shneiderman.
\newblock Flowchart techniques for structured programming.
\newblock {\em SIGPLAN Not.}, 8(8):12--26, 1973.}
\vspace*{\baselineskip}
\endgroup

Dies entspricht dem \emph{plain}-Stil von \BibTeX, beziehungsweise dem
\emph{numeric-comp}-Stil mit Namensabkürzungen in Biblatex. In jedem Fall ist
der Stil schon vorausgewählt. Im Zweifel ist die Variante nach Biblatex
maßgeblich.

\remark{Weder Zitier"~ noch Literaturstil dürfen verändert werden.}

Obwohl \BibTeX gut funktionieren sollte und auch unterstützt wird, wird
Biblatex, insbesondere mit dem \emph{biber}-Backend empfohlen, da dies eine
vernünftige Verwendung von Unicodezeichen und "~schriften erlaubt.
Sammlungsartige Berichte \emph{müssen} Biblatex benutzen. Eine Liste aller
von \BibTeX und Biblatex unterstützten zu zitierenden Werkarten findet sich
in \autoref{tbl:bibtytpes}. Nicht aufgeführte Werkarten erhalten den Typ
\emph{misc}.
\end{ger}
\end{paracol}

\begin{table}
  \makeatletter
  \let\sellang\select@language
  \makeatother
  \centering
  \begin{threeparttable}
    \caption{%
    \textenglish{Supported bibliography entry types} / %
    \textgerman{\em Unterstützte Bibliographieeintragsarten}
    \hspace{\textwidth}\relax D %^^A I am a sacrifice Oo
    \textenglish{Default entry type is indicated; alias entry types with emphasis} /
    \textgerman{\em Die Standardeintragsart ist markiert; Parallelbezeichnungen
    für Eintragsarten hervorgehoben}
    }
    \label{tbl:bibtytpes}
    \begin{tabular}{>{\relax}l>{\relax}l>{\sellang{english}\RaggedRight}p{.34\linewidth}>{\sellang{ngerman}\em\RaggedRight}p{.34\linewidth}}
    \toprule
    \BibTeX           & Biblatex                   &                      & \\
    \midrule
    article           & article                    & articles in journal, magazine \&c & Zeitschriften"~, Magazin"~ und andere Artikel \\
                      & \emph{suppperiodical}      & supplemental material in periodicals & Zusatzmaterialien in Periodika \\
                      & \emph{review}              & reviews of other works & Rezensionen anderer Arbeiten \\
    book              & book                       & books & Bücher\\
                      & \emph{mvbook}              & multi-volume --- & Mehrbändige --- \\
    booklet           & booklet                    & book-like work without publisher & Buchartiges nicht-verlegtes Werk \\
                      & collection                 & collection of multiple independent contributions &  Sammlung eigenständiger Werke \\
                      & \emph{mvcollection}        & multi-volume --- & Mehrbändige --- \\
                      & \emph{reference}           & references such as dictionaries & Nachschlagewerke wie Wörterbücher\\
                      & \emph{mvreference}         & multi-volume --- & Mehrbändige --- \\
    inbook            & inbook\tnote{a}            & independent part of a book & Eigenständiger Buchteil \\
                      & \emph{bookinbook}          & part of a book that originally was a book & Buchteil, der schon als eigenständiges Buch verlegt wurde \\
                      & \emph{suppbook}            & supplemental material in books & Zusatzmaterialien in Büchern \\
    incollection      & incollection               & independent contribution to a collection & Eigenständiges Werk, das einer Sammlung ist \\
                      & \emph{suppcollection}      & supplemental material in a collection & Zusatzmaterialien in Sammlungen \\
                      & \emph{inreference}         & articles in works of references & Artikel in Nachschlagewerken\\
    inproceedings     & inproceedings              & articles in conference proceedings & Tagungsbandartikel\\
    \emph{conference} & \emph{conference}          & --- ditto & --- dito \\
    manual            & manual                     & documentation works & Handbücher \\
    mastersthesis     & mastersthesis\tnote{b}     & Master’s theses & Master"~ oder Magisterarbeiten \\
\llap{☞} misc        & misc                        & anything (default) & sonstiges (Vorgabewert)\\
                      & online                     & online resource like web sites & Internetquellen wie Webseiten \\
                      & \emph{www}                 & --- ditto & --- dito \\
                      & \emph{electronic}          & --- ditto & --- dito \\
    phdthesis         & phdthesis\tnote{b}         & doctoral dissertation & Dissertation \\
                      & patent                     & patent entries & Patente \\
                      & periodical                 & complete preriodical issue & Ganze Zeitschriftenausgabe \\
    proceedings       & proceedings                & conference proceeding & Konferenz"~ und Tagungsbändge \\
                      & \emph{mvproceedings}       & multi-volume --- & Mehrbändige --- \\
                      & report                     & institutionally published report & Von Institutionen veröffentlichte Berichte\\
    techreport        & \emph{techreport}\tnote{c} & technical reports & Technische Berichte\\
                      & thesis                     & theses submitted to educational institutins & Abschlussarbeiten \\
                      & \emph{mastersthesis}       & (see above) & (siehe oben) \\
                      & \emph{phdthesis}           & (see above) & (siehe oben) \\
    unpublished       & unpublished                & works not formally published & Nicht offiziell veröffentlichte Werke \\
    \bottomrule
    \end{tabular}
    \begin{tablenotes}
    \item [a] Behaves differently than \BibTeX / \emph{Verhält sich anders
    als in \BibTeX}
    \item [b] Actually a subtype of thesis / \emph{Eigentlich eine thesis-Unterart}
    \item [c] Actually a subtype of report / \emph{Eigentlich eine report-Unterart}
    \end{tablenotes}
  \end{threeparttable}
\end{table}
\begin{paracol}{2}
\begin{eng}
\section{Further reading}
\label{sec:reading}
\begin{itemize}
\item Abbreviations (ie ...)
\item 
\end{itemize}
\end{eng}
\begin{ger}
\section{Lesehinweise}
test
\end{ger}
\end{paracol}
%</manual>
\DocInput{hpitr.dtx}
\end{document}
%</!README>
%    \end{macrocode}
% \iffalse meta-comment
%</doc|manual|README>
% \fi^^A meta-comment
%
% \Finale
%
\endinput

%% Local Variables:
%% TeX-master: t
%% End:
